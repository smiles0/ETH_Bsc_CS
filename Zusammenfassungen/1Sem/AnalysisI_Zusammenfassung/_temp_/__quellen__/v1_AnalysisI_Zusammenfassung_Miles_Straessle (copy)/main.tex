
\documentclass[11pt]{article}
\usepackage[left=1mm, right=1mm, top=1mm, bottom=1mm,paperwidth=140mm, paperheight=297mm]{geometry}
\usepackage[utf8x]{inputenc}
\usepackage{amssymb, amsfonts, amsmath}
\usepackage{xcolor}
\usepackage[e]{esvect}
\usepackage{floatrow} % allows to insert pictures into proofs and theorems
\usepackage{cancel}
\usepackage{graphicx}
%\usepackage{picins}
\usepackage{tabularx}
\usepackage{mwe} % for blindtext and example-image-a in example
\usepackage{wrapfig}
\usepackage{framed}
\usepackage{booktabs}
\colorlet{shadecolor}{orange!15}
%\usepackage{multirow}
%\usepackage{multicol}
\columnsep24pt
\columnseprule0.1pt
\usepackage[english,german]{babel} 
\usepackage{array}
\usepackage{enumitem}
\usepackage{float}
\usepackage{cancel}
\usepackage{booktabs}

%%Neudefinition / Vereinfachungen
\newcommand*\conj[1]{\overline{#1}}
\newcommand*\abs[1]{\vert #1 \vert}
\newcommand*\floor[1]{\lfloor #1 \rfloor}
\newcommand*\set[1]{\lbrace #1 \rbrace}
\newcommand{\eqdef}{\xlongequal{\text{def}}}
\newcommand{\lims}{\lim_{x \rightarrow x_0}}
\newcommand*\seq[1]{(#1)_{n = 0}^{\infty}}
\newcommand{\sinx}{sin(x)}
\newcommand{\siny}{sin(y)}
\newcommand{\cosx}{cos(x)}
\newcommand{\cosy}{cos(y)}
\newcommand{\tanx}{tan(x)}
\newcommand{\tany}{tan(y)}

\setlength{\parindent}{0px}
\setlength{\parskip}{5px}
\setlength{\parsep}{0px}

% use greek letters for phi and epsilon
\renewcommand{\phi}{\varphi}
\renewcommand{\epsilon}{\varepsilon}

% bolds math symbols
\newcommand{\bs}{\boldsymbol}
% Shortcut to write caligraphic math symbols
\newcommand{\mc}{\mathcal}
% norm
\newcommand{\norm}[1]{\left| \!\:\! \left| #1 \right| \!\:\! \right|}

% some shortcuts
\newcommand{\ds}{\displaystyle}
\newcommand{\arr}{\rightarrow}
\newcommand{\Arr}{\Rightarrow}
\newcommand{\LRA}{\Leftrightarrow}
\newcommand{\LLRA}{\Longleftrightarrow}
\newcommand{\nop}[1]{}
\newcommand{\rank}{\operatorname{rank}}
\newcommand{\cond}{\operatorname{cond}}
\newcommand{\grad}{\operatorname{grad}}
\newcommand{\argmin}{\mathop{\mathrm{argmin}}}
\newcommand{\argmax}{\mathop{\mathrm{argmax}}}
\newcommand{\mx}{\mathop{\mathrm{max}}}
\newcommand{\bigcupdot}{\bigcup \hspace{-0.35cm} \cdot}

\newcommand*\conj[1]{\overline{#1}}
\newcommand*\abs[1]{\vert #1 \vert}
\newcommand*\floor[1]{\lfloor #1 \rfloor}
\newcommand*\set[1]{\lbrace #1 \rbrace}
\newcommand{\eqdef}{\xlongequal{\text{def}}}
\newcommand{\lims}{\lim_{x \rightarrow x_0}}
\newcommand*\seq[1]{(#1)_{n = 0}^{\infty}}
\newcommand{\sinx}{sin(x)}
\newcommand{\siny}{sin(y)}
\newcommand{\cosx}{cos(x)}
\newcommand{\cosy}{cos(y)}
\newcommand{\tanx}{tan(x)}
\newcommand{\tany}{tan(y)}

% stuff for integrals
\newcommand{\intl}{\int\limits}
\newcommand{\rmd}{\mathrm{d}}
\newcommand{\rmD}{\mathrm{D}}

% number sets
\newcommand{\R}{\mathbb{R}}
\newcommand{\E}{\mathbb{E}}
\newcommand{\Z}{\mathbb{Z}}
\newcommand{\N}{\mathbb{N}}
\newcommand{\Q}{\mathbb{Q}}
\newcommand{\C}{\mathbb{C}}
\newcommand{\K}{\mathbb{K}}
\newcommand{\M}{\mathbb{M}}

% big-o notation
\newcommand{\bigO}{\mathcal{O}}



%%%From other styles
\definecolor{formula}{RGB}{229,204,255}
%%Simon-Color
%\definecolor{formula}{RGB}{153,255,153}

\newcommand{\eqbox}[1]{\fcolorbox{black}{formula}{\hspace{0.5em}$\displaystyle#1$\hspace{0.5em}}}


% matrix
\newcommand{\MATR}[1]{ \displaystyle \left( \begin{matrix} #1 \end{matrix} \right)}

% matrix without braces
\newcommand{\BMATR}[1]{ \displaystyle \begin{matrix} #1 \end{matrix}}

% gauss schema
\newcommand{\ARRA}[2]{ \displaystyle \left( \begin{array}{#1} #2 \end{array} \right)}

% determinante
\newcommand{\DET}[1]{\text{det} \left( #1 \right)}
\newcommand{\DETT}{\text{det}}

% determinante
\newcommand{\GDW}{\Longleftrightarrow}


%macro for vectors 
\newcommand{\vect}[1]{\boldsymbol{#1}}
%%%END From other styles

% 'with' in set notation
\newcommand{\with}{\;|\;}

% hyperref
\usepackage[colorlinks=false,pdfborder = {0 0 0 0}]{hyperref}


\colorlet{shadecolor}{orange!15}
\columnsep24pt
\columnseprule0.1pt

\setlength{\parindent}{0px}
\setlength{\parskip}{5px}
\setlength{\parsep}{0px}
\setcounter{secnumdepth}{4}

%% Redefine the \paragraph command:
\makeatletter
\renewcommand\paragraph{\@startsection{paragraph}{4}{0mm}%
	{-\baselineskip}%
	{0.5\baselineskip}%
	{\normalfont\bfseries}%
}%
\makeatother 

% algorithms
\usepackage{algorithmic}
\usepackage{algorithm}
\algsetup{linenodelimiter=}

% listings
\definecolor{darkgreen}{RGB}{0,127,14}
\definecolor{purple}{RGB}{75,0,130}
\usepackage{listings}


\usepackage{listings}
\usepackage{xcolor}
\lstset { %
    language=Pascal,
    backgroundcolor=\color{black!5}, % set backgroundcolor
    basicstyle=\footnotesize,% basic font setting
    escapeinside={!}{!},
    tabsize=2,
}

\def\doubleunderline#1{\underline{\underline{#1}}}

% theorem package
\usepackage{amsthm}
\usepackage{thmtools}
\newtheoremstyle{my-thm-style}% name
{6pt}% Space above
{5pt}% Space below
{}% Body font
{}% Indent amount
{\bfseries}% Theorem head font
{}% Punctuation after theorem head
{3pt}% Space after theorem head
{}% Theorem head spec (can be left empty, meaning `normal')

\definecolor{grey}{RGB}{40, 55, 71}
\definecolor{blue}{RGB}{93, 173, 226}
\definecolor{green}{RGB}{130, 224, 170}
\definecolor{red}{RGB}{229, 115, 115 }

\declaretheoremstyle[
headfont=\normalfont\bfseries,
notefont=\mdseries, notebraces={(}{)},
bodyfont=\normalfont,
postheadspace=8pt,
spaceabove=1pt,
mdframed={
  skipabove=2pt,
  skipbelow=2pt,
  hidealllines=false,
  backgroundcolor={blue!10},
  innerleftmargin=2pt,
  innerrightmargin=2pt}
]{def}

\declaretheoremstyle[
headfont=\normalfont\bfseries,
notefont=\mdseries, notebraces={(}{)},
bodyfont=\normalfont,
postheadspace=8pt,
spaceabove=1pt,
mdframed={
  skipabove=2pt,
  skipbelow=2pt,
  hidealllines=false,
  backgroundcolor={grey!10},
  innerleftmargin=2pt,
  innerrightmargin=2pt}
]{sat}

\declaretheoremstyle[
headfont=\normalfont\bfseries,
notefont=\mdseries, notebraces={(}{)},
bodyfont=\normalfont,
postheadspace=8pt,
spaceabove=1pt,
mdframed={
  skipabove=2pt,
  skipbelow=2pt,
  hidealllines=false,
  backgroundcolor={green!10},
  innerleftmargin=2pt,
  innerrightmargin=2pt}
]{lem}

\declaretheoremstyle[
headfont=\normalfont\bfseries,
notefont=\mdseries, notebraces={(}{)},
bodyfont=\normalfont,
postheadspace=8pt,
spaceabove=1pt,
mdframed={
  skipabove=2pt,
  skipbelow=2pt,
  hidealllines=false,
  backgroundcolor={green!0},
  innerleftmargin=2pt,
  innerrightmargin=2pt}
]{kor}

\declaretheorem[style=def, name=Def. , numberwithin=section]{definition}
\declaretheorem[style=sat, name=Satz, numberwithin=section]{satz}
\declaretheorem[style=lem, name=Lemma, numberwithin=section]{lemma}
\declaretheorem[style=kor, name=Korollar, numberwithin=section]{korollar}



%%%%%
\begin{document}

\title{Analysis I - D-INFK}
\author{Miles Strässle}
\date{\today}
\maketitle

\setcounter{tocdepth}{2}
%\tableofcontents
\setcounter{page}{1}

% Use compiler for 3x1 Format on A4 Page, ask Author
% include the individual chapters
% \input{kap<n>.tex}
%\clearpage


\part{Zusammenfassung}
\input{src/Folgen.tex}
\input{src/Reihen.tex}

%%%%%%%%%%
\newpage
\section{Stetigkeit}

\subsection{Zwischenwertsatz}
Sei $f: [a,b] \to \R$ eine stetige reele Funktion, die auf einem Intervall
definiert ist. Dann existiert zu jedem $u \in [f(a), f(b)]$ (falls $f(a) \leq
f(b)$, sonst $u \in [f(b), f(a)]$) ein $c \in [a,b]$, sodass gilt: $f(c)= u$.

\subsubsection{Beispiel (Fixpunkt)}
Sei $f: [0,1] \to [0,1]$. Zeige: $f$ hat einen Fixpunkt, d.h. es gibt ein $x
\in [0,1]$ derart, dass $f(x) = x$.

Man erzeugt die Funktion $g: [0,1] \to \R, g(x) := f(x) - x$. Es gilt: $f(x) =
x \Leftrightarrow g(x) = 0$, d.h. ein Punkt x ist genau dann ein Fixpunkt von
$f$ wenn er eine Nullstelle von $g$ ist. Es ist zu zeigen, dass $g$ immer eine
Nullstelle auf $[0,1]$ hat. Als Differenz von zwei stetigen Funktionen ist $g$
stetig. Weil ausserdem $f(x) \in [0,1] \; \forall x \in [0,1]$ gilt, ist $g(0)
\geq 0 \geq g(1)$. Da $g$ stetig ist, gibt es daher nach dem Zwischenwertsatz
ein $x \in [0,1]$ mit $g(x) = 0$ und somit gibt es $f(x) = x$.


\subsection{Lipschitz-Stetigkeit}
Es existiert eine Konstante $L\in \mathbb{R}$, sodass:
\begin{equation*}
	|f(x)-f(y)|\leq L|x-y| \quad \forall x,y \in \Omega
\end{equation*}

\emph{Bemerkung:} Ist $f'$ \textbf{auf $\Omega$ beschr{\"a}nkt}, so ist $f$ Lipschitz-stetig. Lipschitz-Stetigkeit impliziert gleichm{\"a}ssige Stetigkeit.

\subsection{Weierstrass-Kriterium}
F{\"u}r alle $\epsilon > 0$ gibt es ein $\delta(\epsilon, a) >0$, sodass f{\"u}r alle $|x-a|<\delta$ gilt:
\begin{equation*}
	|f(x) -f(a)|<\epsilon
\end{equation*}

\subsection{Gleichm{\"a}ssige Stetigkeit}
F{\"u}r alle $\epsilon > 0$ gibt es ein $\delta(\epsilon) >0$, sodass f{\"u}r alle $|x-y|<\delta$ gilt:
\begin{equation*}
	|f(x)-f(y)| < \epsilon
\end{equation*}
\emph{Bemerkung:} Ist $f$ \textbf{stetig und kompakt}, dann ist sie auch gleichm{\"a}ssig stetig.

\subsection{Punktweise Konvergenz}

$f_n(x)$ konvergiert punktweise falls:
\begin{equation*}
	\forall x\in \Omega \quad \lim_{n\rightarrow\infty}f_n(x) = f(x)
\end{equation*}

\subsection{Gleichm{\"a}ssige Konvergenz}

\paragraph{Grundsatz:} Falls eine Folge stetiger Funktionen $f_n$ gleichm{\"a}ssig gegen $f$ konvergiert, muss $f$ stetig sein.\\

$f_n(x)$ konvergiert gleichm{\"a}ssig falls:
\begin{equation*}
	\lim_{n\rightarrow\infty} \sup|f_n(x) - f(x)| = 0
\end{equation*}

\emph{Bemerkung:} Gleichm{\"a}ssige Konvergenz impliziert punktweise Konvergenz.\\
\underline{Rezpet f{\"u}r gleichm{\"a}ssige Konvergenz} 
\begin{enumerate}[label=(\roman*), noitemsep,topsep=0pt]
	\item Punktweiser Limes berechnen
	\[
	\lim_{n\rightarrow\infty}f_n(x) = f(x) \text{\quad = Grenzfunktion}
	\]
	\item Supremum bestimmen \\
	(Ableitung von $f_n(x)$  oder Absch{\"a}tzung benutzen)
	\[
	\sup|f_n(x) - f(x)|
	\]
	\item Limes $n \to \infty$ bestimmen (vgl. Kriterium glm. Konvergenz)
	\[
	\lim_{n\rightarrow\infty} \sup|f_n(x) - f(x)|
	\]
	Limes = 0 $\rightarrow$ Glm. konvergent mit Grenzfunktion f(x)
	\item Indirekte Methode \\
	\begin{itemize}
		\item $f(x)$ unstetig  auf $\Omega \Rightarrow$ keine glm. Konvergenz	
		\item $f(x)$ stetig, $f_n(x) \leq f_{n+1}(x) \forall x \in \Omega $ und $\Omega$ kompakt $\Rightarrow$ Glm. Konvergenz
	\end{itemize}
	
\end{enumerate}

\begin{equation*}
	\begin{split}
		\text{gegeben:} \quad & f_n:[0,1] \mapsto \mathbb{R},\ f_n(x) = (1 - x^2)x^n \\
		\textbf{punktweisen Limes berechnen:} \quad & \lim_{n \to \infty} (1 - x^2)x^n = 0 \equiv f(x) \\
		& \Rightarrow\ \text{konvergiert gegen 0} \\
		\text{Supremum berechnen:} \quad & \sup_{x \in [0,1]} |f_n(x) -f(x)| = \sup_{x \in [0,1]} |f_n(x)| = \sup_{x \in [0,1]}f_n(x) \\
		\text{Maximum finden:} \quad & \frac{d}{dx} f_n(x) nx^{n-1}(1-x^2)-2xx^n = x^{n-1}(n - (n+2)x^2) = 0 \\
		& \Rightarrow x_1 = 0,\ x_2 = \sqrt{\frac{n}{n+2}} \Rightarrow x_2\ \text{ist Maximum} \\
		\textbf{Limes berechnen:} \quad & \lim_{n \to \infty} \sup_{x \in [0,1]} |f_n(x) - f(x)| = \lim_{n \to \infty} f_n(x_2) = 0 \\
		\textbf{Folgerung:} \quad & \text{$f_n$ konvergiert auf $[0,1]$ glm. gegen $f$}
	\end{split}
\end{equation*}
 % (ZWS, MINMAX, Umkehrabb, Konvergenz)

\section{Grenzwert}

\subsection{Dominanz}

\begin{equation*}
	\begin{split}
		\text{F{\"u}r}\ x \to +\infty:\quad & ... < \log(\log(x)) < \log(x) < x^\alpha < \alpha^x < x! < x^x \\
		\text{F{\"u}r}\ x \to 0:\quad & ... < \log(\log(x)) < \log(x) < (\frac{1}{x})^\alpha \\
	\end{split}
\end{equation*}

\subsection{Fundamentallimes}

\begin{equation*}
	\begin{split}
		\lim_{x \to a} \frac{\sin \odot}{\odot} = \lim_{x \to a} \frac{\tan \odot}{\odot} & = 1\ \text{mit}\ \odot \xrightarrow{\: x \to a \: } 0 \\ 
		\lim_{x \to a} (1 + \frac{1}{\odot})^\odot & = e\ \text{mit}\ \odot \xrightarrow{\: x \to a \: } \infty \\ 
		\lim_{x \to a} (1 + \odot)^\frac{1}{\odot} & = e\ \text{mit}\ \odot \xrightarrow{\: x \to a \: } 0 \\ 
	\end{split}
\end{equation*}

\subsection{Wurzeltrick}

\begin{equation*}
	\lim_{x\to\infty} \sqrt{\alpha}+\beta = \lim_{x\to\infty}(\sqrt{\alpha}+\beta)\frac{\sqrt{\alpha}-\beta}{\sqrt{\alpha}-\beta}
\end{equation*}
\newpage
\subsection{$e^{\log(x)}$-Trick}

\paragraph{Anforderung:}Term der Form $f(x)^{g(x)}$ mit Grenzwert "$0^0$", "$\infty^0$" oder "$1^\infty$" f{\"u}r $x \to 0$

\begin{equation*}
	\textbf{Grundsatz:}\quad\lim_{x\to a}f(x)^{g(x)} = \lim_{x\to a}e^{g(x) \cdot \log(f(x))}
\end{equation*}

\emph{Tipp:} Danach den Limes des Exponenten berechnen. Oft ist Bernoulli-de l'H{\^o}pital dazu n{\"u}tzlich.

\subsection{Substitution}

\begin{equation*}
	\begin{split}
		\lim_{x\to \infty} x^2(1 - \cos(\frac{1}{x})) \Rightarrow u = \frac{1}{x} \Rightarrow \lim_{x\to 0} \frac{1 - \cos(u)}{u^2}
	\end{split}
\end{equation*}

\subsection{Satz von Bernoulli-de l'H{\^o}pital}

\paragraph{Anforderung:}Term der Form $\frac{f(x)}{g(x)}$ mit Grenzwert entweder "$\frac{0}{0}$" oder "$\frac{\infty}{\infty}$" mit $g'(x) \neq 0$. \\

\begin{equation*}
	\textbf{Grundsatz:}\quad\lim_{x\to a}\frac{f(x)}{g(x)} = \lim_{x\to a}\frac{f'(x)}{g'(x)}
\end{equation*}

\begin{table}[H]
	\centering
	\begin{tabular}{|c|c|c|}
		\hline
		\textbf{Term} & \textbf{Anforderung} & \textbf{Umformung} \\ \hline
		$f(x)g(x)$              & "$0\cdot\infty$"                     & $\frac{g(x)}{\frac{1}{f(x)}}$          \\ \hline 
		$\frac{f(x)}{g(x)} - \frac{h(x)}{i(x)}$ & "$\infty - \infty$"  & $\frac{f(x)i(x) - h(x)g(x)}{g(x)i(x)}$ \\ \hline      
	\end{tabular}
\end{table}


\section{Differenzialrechnung}

Eine stetige Funktion ist differenzierbar, falls der Grenzwert $f'(x_0)$ existiert:

\begin{equation*}
f'(x_0) := \lim_{x\to x_0}\frac{f(x) - f(x_0)}{x-x_0}
\end{equation*}
\subsubsection*{Tangente}
Sei $f: \Omega \to \mathbb{R}$ an der Stelle $x_0 \in \Omega$ diffbar. Dann ist die Tangente im Punkt $x_0$
\[ t(x; x_0)=f(x_0)+f'(x_0)(x-x_0) \]

\subsection{Umkehrsatz}

\begin{equation*}
(f^{-1})'(y) = \frac{1}{f'(f^{-1}(y))}
\end{equation*}

\subsection{Mittelwertsatz}

\begin{equation*}
f'(c) = \frac{f(b) - f(a)}{b - a}
\end{equation*}

\subsection{Taylorpolynom}

Das Taylorpolynom $m$-ter Ordnung von $f(x)$ an der Stelle $x=a$
\begin{equation*}
P^a_m(x) := f(a) + f'(a)(x-a) + \frac{1}{2}f''(a)(x-a)^2 + ... + \frac{1}{m!} f^{(m)}(a)(x-a)^m
\end{equation*}

mit dem Fehlerterm $R^a_m(x)$, wobei $\xi$ zwischen $a$ und $b$ liegt:
\begin{equation*}
R^a_m(x) = \frac{f^{(m+1)}(\xi)}{(m+1)!}(x+a)^{m+1},\ \text{wobei}\ f(x) = P^a_m(x) + R^a_m(x)
\end{equation*}

\section{Vollständige Induktion}\index{Induktion}
Grundlägende Struktur um die Aussage $A(n)$ zu beweisen:
\begin{enumerate}
	\item \textbf{Induktionsanfang/Verankerung:} Die Aussage wird für $n = A$ bewiesen.
	$A$ ist dabei meistens der erste Wert für die gegebene Eingabemenge.
	Der Beweis wird meist durch direktes ausrechnen gemacht.
	\item \textbf{Annahme/Induktionsvoraussetzung:} Hier schreibt man,
	dass man davon ausgeht die Aussage sei gültig (damit man sie im nächsten Schritt)
	einsetzen kann. Man kopiert also im Grunde, was man zu beweisen hat mit einigen Zierwörter.
	\item \textbf{Induktionsschritt:} Für jedes $n \geq A$ wird unter Benutzung der Aussage $A(n)$
	die Aussage $A(n+1)$ bewiesen. Dazu wird die Induktionsannahme verwendet.
\end{enumerate}

\subsection{Beispiel}
		Es ist zu beweisen, dass für jedes $n \in \N$ folgendes gilt: $1 + 2 + 3 + \ldots + n = \frac{n(n + 1)}{2}$		
		
		Lemma: $\forall \in \N. 1 + 2 + 3 + \ldots + n = \frac{n(n + 1)}{2}$ \\
		
	Beweis: \\
		  Sei $P(n) \equiv 1 + 2 + 3 + \ldots + n = \frac{n(n + 1)}{2}$. \\
		  Wir zeigen $\forall \in \N. P(n)$ mit vollständiger Induktion.\\
		
		Induktionsanfang: Zeige $P(0)$. \\
		 $$0 =  \frac{0(0 + 1)}{2}$$
		
		  Induktionsschritt: \\
		    Sei $n \in N$ beliebig und nehmen wir $P(n)$ an (Induktionsvoraussetzung).  \\
		    Zeige $P(n+1)$ (Induktionsbehauptung).\\
		  \begin{align}
		      1 + 2 + 3 + \ldots + n + (n + 1) =\\
		      = \frac{n(n + 1)}{2} + (n+1)       & \text{ — P(n) Induktionsvor. }\\
		      = \frac{n(n + 1)}{2} + \frac{2(n + 1)}{2}  & \text{—arith}\\
		      = \frac{(n+1)*((n+1)+1)}{2}        & \text{—arith}
		\end{align}
		qed.

\subsection{Hauptsatz der Differential- und Integralrechnung}

\begin{equation*}
f(x)=\int^{m(x)}_lg(t)dt
\end{equation*}
\begin{equation*}
f'(x)=g(m(x))\cdot\frac{d}{dx}m(x)
\end{equation*}
wobei $m(x)$ der Form $ax^b$ ist mit $l\in \mathbb{R}$\\\\
\underline{Differenzial / Jaccobi-Matrix}
\begin{equation*}
df = 
\begin{pmatrix}
\frac{\partial f_1}{\partial x_1} & ... & \frac{\partial f_1}{\partial x_n} \\
... & ... & ... \\
\frac{\partial f_m}{\partial x_1} & ... & \frac{\partial f_m}{\partial x_n}
\end{pmatrix}
\end{equation*}
$\rightarrow$ enth{\"a}lt die $n$ part. Ableit. aller m Komponenten von $f$\\
\subsubsection*{Taylorentwicklung mit mehreren Variabeln} 
\begin{tabular}{ll}
	$f(x,y) =  $ & $f(x_0,y_0) + \frac{\partial f}{\partial x} \Delta x + \frac{\partial f}{\partial y} \Delta y$ \\
	& $ + \frac{1}{2!} \bigg{(}
	\frac{\partial^2 f}{\partial x^2} (\Delta x)^2	
	+ 2\frac{\partial^2 f}{\partial x \partial y} \Delta x \Delta y 
	+ \frac{\partial^2 f}{\partial y^2} (\Delta y)^2	
	\bigg{)} $ \\
	& $ + \frac{1}{3!} \bigg{(}
	\frac{\partial^3 f}{\partial x^3} (\Delta x)^3
	+ 3\frac{\partial^3 f}{\partial x^2 \partial y} (\Delta x)^2 \Delta y$ \\
	&			\qquad  $+ 3\frac{\partial^3 f}{\partial x \partial y^2} \Delta x (\Delta y)^2 
	+ \frac{\partial^3 f}{\partial y^3} (\Delta y)^3
	\bigg{)} $ \\
	& 			$+ \cdots$			
	
\end{tabular}

\section{Integration}


\subsection{Regeln}

\begin{equation*}
\begin{split}
\textbf{Direkter Integral}\quad & \int f(g(x))g'(x)\ dx = F(g(x)) \\
\textbf{Partielle Integration}\quad & \int f' \cdot g\ dx = f \cdot g - \int f \cdot g'\ dx \\
\textbf{mit Polynomen}\quad & \int\frac{p(x)}{q(x)}\ dx \Rightarrow\ \text{Partialbruchzerlegung} \\
\textbf{Substitution}\quad & \int_a^b f(\varphi(t))\varphi'(t)\ dt = \int_{\varphi(a)}^{\varphi(b)} f(x)\ dx\ \text{mit}\ x = \varphi(t)
\end{split}
\end{equation*}

\subsection{Tipps}

\begin{equation*}
\begin{split}
\int\tan x\ dx & = \int\frac{\sin x}{\cos x}\ dx = -\log|\cos(x)| \\
\int \frac{1}{x - \alpha}\ dx & = \log(x-\alpha) \\
\int\frac{\frac{1}{\alpha}}{1+(\frac{x}{\alpha})^2}\ dx & = \arctan(x) \\
\int \sin^2(x)\ dx & = \frac{1}{2}(x - \sin(x)\cos(x)) + C \\
\int \cos^2(x)\ dx & = \frac{1}{2}(x + \sin(x)\cos(x)) + C \\
\int \sqrt{x^2+1}\ dx & = \sinh(x) + C
\end{split}
\end{equation*}

\subsection{Uneigentliche Integrale}

\begin{equation*}
\begin{split}
\int_0^\infty f(x)\ dx = \lim_{R \to \infty} \int_0^R f(x)\ dx \\
\int_{-\infty}^\infty f(x)\ dx = \lim_{R \to -\infty} \int_R^k f(x)\ dx + \lim_{R \to \infty} \int_k^R f(x)\ dx
\end{split}
\end{equation*}

Gibt es eine Unstetigkeitstelle $c$ in dem Integrationsgebiet, so geht man wie folgt vor:
\begin{equation*}
\int_a^b f(x)\ dx = \lim_{\varepsilon \to 0} \int_a^{c-\varepsilon} f(x)\ dx + \lim_{\varepsilon \to 0} \int_{c+\varepsilon}^b f(x)\ dx
\end{equation*}

\subsection{Beweis bijektiver Funktionen}

Zu beweisen sind folgende Eigenschaften:
\begin{description}[labelindent=16pt,style=multiline,leftmargin=3cm, noitemsep]
	\item[injektiv:] Zeig, dass $f$ \textbf{strickt monoton w{\"a}chst oder f{\"a}llt} und\textbf{stetig} ist
	\item[surjektiv:] Zeig, dass alle Werte im Bildbereich angenommen werden (vl. mit Zwischenwertsatz)
\end{description}
Daraus folgt dann, dass $f$ bijektiv ist.


\section{Differentialgleichungen}

\subsection{Grundbegriffe}

\begin{description}[labelindent=16pt,style=multiline,leftmargin=3.5cm, noitemsep]
	\item[Ordnung:] h{\"o}chste vorkommende Ableitung
	\item[linear:] alle $y$-abh{\"a}ngigen Terme kommen linear vor (keine Terme wie zum Beispiel $y^2$, $(y'')^3$, $\sin(y)$, $e^{y'}$)
	\item[homogen:] Gleichung ohne St{\"o}rfunktionen
	\item[St{\"o}rfunktion:] Term, der rein von der Funktionsvariablen $x$ abh{\"a}ngt
\end{description}

\subsection{Methoden}

\begin{table}[H]
	\centering
	\begin{tabular}{|p{3cm}|p{6cm}|p{3cm}|}
		\hline
		& \textbf{Problem} 							& \textbf{Anforderungen} 			\\ \hline
		\textbf{Trennung der Variablen}   	& $y' = \frac{dy}{dx} = h(x) \cdot g(y)$ 	& 1. Ordnung			            \\ \hline
		\textbf{Variation der Konstanten}	& $y' = \frac{dy}{dx} = h(x)y + b(x)$	 	& 1. Ordnung \filbreak inhomogen	\\ \hline
		\textbf{Euler-Ansatz}				& $a_{n}y^{(n)} + a_{n-1}y^{(n-1)} + ... + a_{0}y = 0$	 	& n. Ordnung \filbreak linear \filbreak homogen	\\ \hline
		\textbf{Direkter Ansatz}				& $a_{n}y^{(n)} + a_{n-1}y^{(n-1)} + ... + a_{0}y = b(x)$	& n. Ordnung \filbreak linear \filbreak inhomogen	\\ \hline
		
		%\textbf{Substitution}				& $y' = h(\frac{y}{x})$ \filbreak $y' = h(ax + by + c)$ \filbreak $y' = h(\frac{ax + by + c}{dx + ey + f})$ \filbreak $y' = \frac{y}{x}h(xy)$ 																		& nicht direkt separierbar			\\ \hline
	\end{tabular}
\end{table}

\subsubsection{Trennung der Variable}

\begin{equation*}
\begin{split}
& y' + x \tan y = 0,\ y(0) = \frac{\pi}{2} \\
\text{umformen}\quad & \frac{dy}{dx} = -x \tan y \\
\textbf{konstante L{\"o}sungen}\quad & y(x) \equiv 0\ \text{erf{\"u}llt jedoch $y(0) \equiv \frac{\pi}{2}$ nicht} \\
\text{Trennung}\quad & \frac{dy}{\tan y} = -x dx \\
\text{integrieren}\quad & \int\frac{\cos y}{\sin y}dy = - \int xdx \Rightarrow \log|\sin y| = -\frac{x^2}{2} + C \\
& \Rightarrow |\sin y| = e^Ce^{\frac{-x^2}{2}} \Rightarrow \sin y = \pm e^Ce^{\frac{-x^2}{2}} = Ce^{\frac{-x^2}{2}} \\
\text{Anfangsbedingung gebrauchen}\quad & \sin(y(0)) = \sin (\frac{\pi}{2}) = 1 \Rightarrow C = 1 \\
\textbf{L{\"o}sung}\quad & y(x) = \arcsin (e^{\frac{-x^2}{2}})
\end{split}
\end{equation*}



\subsubsection{Variation der Konstanten}

\begin{equation*}
\begin{split}
\textbf{Grundsatz:} &\quad y(x) = y_h(x) + y_p(x) \\
& y'(x+1) + y = x^3,\ y(0) = \sqrt{5} \\
\text{Trennung}\quad & \frac{y'}{y} = \frac{-1}{x+1} \\
\textbf{konstante L{\"o}sungen}\quad & y(x) \equiv 0\ \text{erf{\"u}llt jedoch $y(0) \equiv \sqrt{5}$ nicht} \\
\text{integrieren}\quad & \int \frac{dy}{y} = - \int \frac{dx}{x+1} \\
& \Rightarrow \ln|y| = -\ln|x+1| + C \\
\end{split}
\end{equation*}
\begin{equation*}
\begin{split}
\textbf{Homogene L{\"o}sung} \quad & y_h(x) = \frac{C}{x+1},\ \text{mit}\ C= \pm e^C \in \mathbb{R}\backslash{0} \\
\text{partikul{\"a}rer Ansatz}\quad & y_p(x) = \frac{C(x)}{x+1} \\
\text{einsetzen} \quad & (\frac{C'(x)}{x+1} - \frac{C(x)}{(x+1)^2})(x+1) + \frac{C(x)}{x+1} = x^3 \\
& C'(x) = x^3 \\
& C(x) = \frac{x^4}{4} \\
\textbf{partkul{\"a}re L{\"o}sung} \quad & y_p(x) = \frac{x^4}{4(x+1)} \\
\text{allgemeine L{\"o}sung}\quad & y(x) = y_h(x) + y_p(x) = \frac{C}{x+1} + \frac{x^4}{4(x+1)} \\
\text{Anfangsbedingung benutzen} \quad & y(0) = \sqrt{5} \Rightarrow C = \sqrt{5} \\
\textbf{L{\"o}sung} \quad & y(x) = \frac{\sqrt{5}}{x+1} + \frac{x^4}{4(x+1)}
\end{split}
\end{equation*}

%\subsubsection{Substitution}
%
%\begin{equation*}
%\begin{split}
%	& y' = h(\frac{y}{x})\ \text{ersetzt durch}\ z(x) = \frac{y(x)}{x} \Leftrightarrow y(x) = xz(x) \\
%	& \Rightarrow	y' = z + xz'
%\end{split}
%\end{equation*}



\subsubsection{Euler-Ansatz}

\begin{equation*}
\begin{split}
& y'' - 2y' - 8y = 0,\ y(1) = 1, y'(1) = 0 \\
\text{Euler-Ansatz}\quad & y(x) = e^{\lambda x} \\
\text{einsetzen}\quad & \lambda^2 e^{\lambda x} - 2\lambda e^{\lambda x} - 8e^{\lambda x} = 0 \\
\textbf{charakt. Polynom}\quad & \lambda^2 - 2\lambda - 8 = (\lambda - 4)(\lambda + 2) = 0 \\
\text{Nullstellen}\quad & 4, -2 \\
\textbf{allgemeine L{\"o}sung}\quad & y(x) = Ae^{4x} + Be^{-2x} \\
\text{Anfangsbedingung gebrauchen}\quad & y(1) = Ae^4 + Be^{-2} =1,\\ &y'(1) = 4Ae^4 - 2Be^{-2} = 0 \\
& \Rightarrow A = \frac{1}{3}e^{-4}, B = \frac{2}{3}e^2 \\
\textbf{L{\"o}sung}\quad & y(x) = \frac{1}{3}e^{4x-4} + \frac{2}{3}e^{2-2x}
\end{split}
\end{equation*}

\emph{Bemerkung:} Zu einer $m$-fachen Nullstelle $\lambda$ geh{\"o}ren die $m$ linear unabh{\"a}ngigen L{\"o}sungen $e^{\lambda x}$, $x\cdot e^{\lambda x}$, ... , $x^{m-1}\cdot e^{\lambda x}$. Zur $m$-fachen Nullstelle $\lambda = 0$ geh{\"o}ren die L{\"o}sungen $1$, $x$, ... , $x^{m-1}$. \\

\emph{Komplexe Nullstellen:}


$x = \frac{-b \pm \sqrt{b^2-4ac}}{2a}$


Ein komplexes Nullstellenpaar der Form $\alpha \pm \beta i$ liefert folgende homogene L{\"o}sung:
\begin{equation*}
y(x)=e^{\alpha x}(C_1\cos(\beta x) + C_2\sin(\beta x))
\end{equation*}



\subsubsection{Direkter Ansatz}

\begin{equation*}
\textbf{Grundsatz:}\quad y(x) = y_\text{homo}(x) + y_p(x)
\end{equation*}

\begin{table}[H]
	\centering
	\begin{tabular}{|l|l|l|}
		\hline
		\textbf{Inhomogener Term $b(x)$} & \textbf{Ansatz f{\"u}r $y_p(x)$}	& \textbf{zu bestimmen}		\\ \hline
		Polynom				& $Ax^2 + Bx + C$			& $A$, $B$, $C$		\\ \hline
		$c e^{k x}$ & $Ae^{kx}$					& $A$				\\ \hline
		$c\sin(kx)$ oder $c\cos(kx)$ & $A\sin(kx) + B\cos(kx)$ & $A$, $B$ \\ \hline
		
	\end{tabular}
\end{table}

\emph{Bemerkung:} Kommt der gew{\"a}hlte Ansatz schon in der homogenen L{\"o}sung vor,\\ so multipliziert man den Ansatz einfach mit $x$.

\begin{equation*}
\begin{split}
& y'' - y' + \frac{1}{4}y = \cos(x) \\
\text{homogener Ansatz}\quad & y'' + y' + \frac{1}{4}y = 0 \\
\text{Euler-Ansatz anwenden}\quad & \lambda^2 + \lambda + \frac{1}{4} = (\lambda + \frac{1}{2})^2 = 0 \\
\textbf{homogene L{\"o}sung}\quad &\Rightarrow y_\text{homo}(x) = Ae^{-\frac{x}{2}} + Bx \cdot e^{-\frac{x}{2}} \\
\text{partikul{\"a}rer Ansatz w{\"a}hlen}\quad & y_p(x) = a\cos(x) + b\sin(x) \\
& \Rightarrow y_p'(x) = -a\sin(x) + b\cos(x),\  y_p''(x) = \\ & = -a\cos(x) -b \sin(x) \\
\text{Einsetzen}\quad & (-a + b + \frac{a}{4})\cos(x) + (-b -a + \frac{1}{4}b)\sin(x) \\&= \cos(x) \\
\text{Koeffizientenvergleich}\quad & -\frac{3}{4}a + b = 1,\ -a-\frac{3}{4}b = 0 \\
\end{split}
\end{equation*}

\begin{equation*}
\begin{split}
\textbf{partikul{\"a}re L{\"o}sung}\quad & y_p(x) = -\frac{12}{25}\cos(x) + \frac{16}{25}\sin(x) \\
\textbf{L{\"o}sung}\quad & y(x) = Ae^{-\frac{x}{2}} + Bx \cdot e^{-\frac{x}{2}} -\frac{12}{25}\cos(x) + \frac{16}{25}\sin(x)
\end{split}
\end{equation*}

\includegraphics[width=1\textwidth]{images/dgl_ansatze_part_lsg.png}

\section{Komplexe Zahlen}


%
\begin{minipage}[c]{0.45\textwidth}
	\begin{equation*}
	\begin{split}
	z & = x + iy = r(\cos(\varphi) + i\sin(\varphi)) = re^{i\varphi} \\
	r & = |z| = \sqrt{x^2 + y^2} \\
	\arg(z) & = \varphi  = \arctan(\frac{y}{x}) \quad \text{(je nach Quadrant)}  \\
	x & = r\cos(\varphi) \\
	y & = r\sin(\varphi) \\
	zw & = (re^{i\varphi})\cdot(se^{i\psi}) = rse^{i(\varphi + \psi)} \\
	\sqrt[q]{z} & = \sqrt[q]{s}e^{i\phi}\text{, wobei }\phi = \frac{\varphi}{q} \mod \frac{2\pi}{q} \\
	e^{i(\frac{\pi}{2} + 2\pi k)} & = i,\ e^{i\pi} = 1, \ e^{-i\pi} = -1
	\end{split}
	\end{equation*}
\end{minipage}

\begin{minipage}[c]{0.5\textwidth}
	\begin{equation*}
	\begin{split}
	(a,b) \cdot (c, d) & = (ac-bd, ad+bc) \\
	\overline{z} & = x - iy\\
	z^{-1} & = \frac{\overline{z}}{|z|^2} \\
	i & = \sqrt{-1}\\
	\end{split}
	\end{equation*}
\end{minipage}
%
\begin{minipage}[c]{0.5\textwidth}
	\begin{equation*}
	\begin{split}
	i^2 & = -1 \\
	|z|^2 & = z\overline{z} \\
	|zw|^2 & = (zw) \cdot \overline{(zw)} = |z|^2|w|^2
	\end{split}
	\end{equation*}
\end{minipage}



\part{Tables}
\subsection{Elementare Integrale}

\begin{table}[H]
	\centering
	\begin{tabular}{|c|c|c|}
		\hline
		$f'(x)$ & $f(x)$ & $F(x)$ \\ \specialrule{.1em}{0em}{0em} 
		$\frac{f'(x)g(x) - f(x)g'(x)}{g(x)^2}$ & $\frac{f(x)}{g(x)}$ &  \\ \hline
		$0$ & $c$ & $cx$ \\ \hline
		$r\cdot x^{r-1}$ & $x^r$ & $\frac{x^{r+1}}{r+1}$ \\ \hline
		$-\frac{1}{x^2} = -x^{-2}$ & $\frac{1}{x} = x^{-1}$ & $\ln|x|$ \\ \hline
		$\frac{1}{2\sqrt{x}} = \frac{1}{2}x^{-\frac{1}{2}}$ & $\sqrt{x} = x^{\frac{1}{2}}$ & $\frac{2}{3}x^\frac{3}{2}$ \\ \hline
		$\cos(x)$ & $\sin(x)$ & $-\cos(x)$ \\ \hline
		$-\sin(x)$ & $\cos(x)$ & $\sin(x)$ \\ \hline
		$1 + \tan^2(x) = \frac{1}{\cos^2(x)}$ & $\tan(x)$ & $-\ln|\cos(x)|$ \\ \hline
		$e^x$ & $e^x$ & $e^x$ \\ \hline
		$c\cdot e^{cx}$ & $e^{cx}$ & $\frac{1}{c}\cdot e^{cx}$ \\ \hline
		$\ln(c)\cdot c^x$ & $c^x$ & $\frac{c^x}{\ln(c)}$ \\ \hline
		$\frac{1}{x}$ & $\ln|x|$ & $x(\ln|x| - 1)$ \\ \hline
		$\frac{1}{\ln(a) \cdot x}$ & $\log_a|x|$ & $\frac{x}{\ln(a)}(\ln|x| -1)$ \\ \hline
		$\frac{1}{\sqrt{1-x^2}}$ & $\arcsin(x)$ & $x\cdot\arcsin(x) + \sqrt{1-x^2}$ \\ \hline
		$-\frac{1}{\sqrt{1-x^2}}$ & $\arccos(x)$ & $x\cdot\arccos(x) - \sqrt{1-x^2}$ \\ \hline
		$\frac{1}{1+x^2}$ & $\arctan(x)$ & $x\cdot \arctan(x) - \frac{1}{2}\ln(1+x^2)$ \\ \hline
		$\cosh(x)$ & $\sinh(x) = \frac{e^x - e^{-x}}{2}$ & $\cosh(x)$ \\ \hline
		$\sinh(x)$ & $\cosh(x) = \frac{e^x + e^{-x}}{2}$ & $\sinh(x)$ \\ \hline
		$\frac{1}{\cosh^2(x)}$ & $\tanh(x)$ & $\log(\cosh(x))$ \\ \hline
	\end{tabular}
\end{table}



\subsection{Wichtige Grenzwerte}

\begin{equation*}
\begin{split}
\lim\limits_{n \to \infty} \left( 1+\frac{x}{n} \right)^n = e^x \qquad & \qquad \lim\limits_{n \to \infty} \left( 1+\frac{1}{n} \right)^n = e \\
\lim\limits_{x \to 0} \frac{a^x-1}{x} = \ln a \qquad & \qquad \lim\limits_{x \to 0} \frac{\log_a(1+x)}{x} = \frac{1}{\ln a} \\
\lim\limits_{x \to 0} \frac{1-\cos(x)}{x} = 0 \quad \qquad & \qquad \lim\limits_{x \to 0} \frac{1-\cos(x)}{x^2} = \frac{1}{2} \\
\lim\limits_{x \to 0} \frac{\tan(x)}{x} = 1 \qquad & \qquad \lim\limits_{x \to 0} \frac{\sin(x)}{x} = 1 \\
\lim\limits_{n \to \infty} \frac{n!}{n^n} = 0 \qquad & \qquad \lim\limits_{n \to 0} \frac{e^n -1 }{n} = 1 \\
\lim\limits_{n \to \infty} \sqrt[n]{n!} = \infty \qquad & \qquad \lim\limits_{n \to \infty} \sqrt[n]{n} = 1 \\
\lim\limits_{n \to \infty} \ln(n) = \infty \qquad & \qquad \lim\limits_{x \to 0} \frac{\log_a(1+x)}{x} = \frac{1}{\ln a} \\
\end{split}
\end{equation*}


\part{Generelles}

\section{Trigonometrische Definitionen \& Sätze}
\subsection{Definitionen}

$sin(x) := \sum_{n=0}^{\infty} (-1)^n \frac{x^{2n+1}}{(2n+1)!} = \frac{x}{1!} - \frac{x^3}{3!} + \frac{x^5}{5!} \mp ...$

$cos(x) := \sum_{n=0}^{\infty} (-1)^n \frac{x^{2n}}{(2n)!} = \frac{x^0}{0!} - \frac{x^2}{2!} + \frac{x^4}{4!} \mp ...$

$exp(x) := \sum_{n=0}^{\infty} \frac{x^{n}}{n!} = \lim\limits_{n \rightarrow \infty}{(1 + \frac{x^{n}}{n})^n}$

$arctan(x) := \sum_{n=0}^{\infty} (-1)^n \frac{x^{2n+1}}{2n+1} = \frac{x}{1} - \frac{x^3}{3} + \frac{x^5}{5} \mp ...$

Hinweis: Für einfache Approximation genügt es die ersten paar Glieder der $arctan(x)-Reihe$ zu berechnen. \newline 
Falls: $x \notin [0,1]$, gibt es eine Vereinfachung:
$ arctan(x) = \frac{sgn(x) * \pi}{2} - arctan(\frac{1}{x}) $

\subsubsection{Definition Taylorreihe} Eine Funktion $f(x)$ wird an einer Stelle $x_0$ angenähert durch $Tf(x;x_0) = \sum_{n=0}^{\infty} \frac{f^{(n)}(x_0)}{n!}(x - x_0)^n = f(x_0)$ \\

\subsubsection{Definitionen csc(x), sec(x), cot(x)}
$csc(x) := \frac{1}{sin(x)}$ \hfil $sec(x) := \frac{1}{cos(x)}$ \hfil $cot(x) := \frac{1}{tan(x)} = \frac{cos(x)}{sin(x)}$





\subsection{Periodizit"aten}

\begin{minipage}{0.49\linewidth}

\begin{itemize}
	\item \(1\cdot e^{2\pi i k} = 1\) f"ur alle \(k \in \mathbb{Z}\)
	\item	\(e^{\frac{\pi}{2} i k} = i\)
	\item \(e^{-\frac{\pi}{2} i k} = -i\)
	\item \(e^{-2 \pi i k} = 1\)
	\item \(e^{\pi i k} = (-1)^k\)
	\item \(e^{-\pi i k} = (-1)^k\)
\end{itemize}	

\end{minipage}
\hfill
\begin{minipage}{0.49\linewidth}
\begin{itemize}	
	\item $ \sin(z+2 \pi) = \sin(z) $
	\item $ \cos(z+2 \pi) = \cos(z) $
	\item $\sinh(z + 2 \pi i) = \sinh(z)$
	\item $\cosh(z + 2 \pi i) = \cosh(z)$
	\item $ \sin(z- \pi) = -\sin(z) $
	\item $ \cos(z- \frac{\pi}{2}) = \sin(z) $
\end{itemize}
\end{minipage}
\subsection{Winkel}
\renewcommand{\arraystretch}{1.5}
\begin{tabular}{|c|c|c|c|c|c|c|c|c|}
	\hline
	\(\varphi \) &$0$ & \(\frac{\pi}{6}\) & \(\frac{\pi}{4}\) & \(\frac{\pi}{3}\) &  \(\frac{\pi}{2}\) &  \(\frac{2\pi}{3}\) &  \(\frac{3\pi}{4}\) & \(\frac{5\pi}{6}\) \\
	\hline
	Grad & $0^\circ$ & $30^\circ$ & $45^\circ$ & $60^\circ$ & $90^\circ$ & $ 120^\circ$ & $135^\circ$ & $150^\circ$\\
	\hline
	\(\sin(\varphi)\) & $0$ & \(\frac{1}{2}\) & \(\frac{\sqrt{2}}{2}\) & \(\frac{\sqrt{3}}{2}\) & $1$ &  \(\frac{\sqrt{3}}{2}\) &  \(\frac{1}{\sqrt{2}}\) &  \(\frac{1}{2}\)   \\
	\hline
	\(\cos(\varphi)\) &$1$ & \(\frac{\sqrt{3}}{2}\) & \(\frac{\sqrt{2}}{2}\) & \(\frac{1}{2}\) & $0$ & $-\dfrac{1}{2}$ & \(-\frac{1}{\sqrt{2}}\) &  \(-\frac{\sqrt{3}}{2}\)  \\
	\hline
	\(\tan(\varphi)\) &$ 0$ & \(\frac{1}{\sqrt{3}}\) &  $1$ & $\sqrt{3}$ & $\pm \infty$ &$-\sqrt{3}$ & $-1$ &  \(-\frac{1}{\sqrt{3}}\)\\ 
	\hline
\end{tabular}\\

\begin{tabular}{|c|c|c|c|c|c|c|c|}
	\hline
	\(\varphi \) &$\pi$ & \(\frac{7\pi}{6}\) & \(\frac{5\pi}{4}\) & \(\frac{4\pi}{3}\) &  \(\frac{3\pi}{2}\) &  \(\frac{5\pi}{3}\) &  \(\frac{7\pi}{4}\)  \\
	\hline
	Grad & $180^\circ$ & $210^\circ$ & $225^\circ$ & $240^\circ$ & $270^\circ$ & $ 300^\circ$ & $315^\circ$\\
	\hline
	\(\sin(\varphi)\) & $0$ & \(-\frac{1}{2}\) & \(-\frac{\sqrt{2}}{2}\) & \(-\frac{\sqrt{3}}{2}\) & $-1$ &  \(-\frac{\sqrt{3}}{2}\) &  \(-\frac{1}{\sqrt{2}}\)  \\
	\hline
	\(\cos(\varphi)\) &$-1$ & \(-\frac{\sqrt{3}}{2}\) & \(-\frac{\sqrt{2}}{2}\) & \(-\frac{1}{2}\) & $0$ & $\dfrac{1}{2}$ & \(\frac{1}{\sqrt{2}}\)  \\
	\hline
	\(\tan(\varphi)\) &$ 0$ & \(\frac{1}{\sqrt{3}}\) &  $1$ & $\sqrt{3}$ & $\pm \infty$ &$-\sqrt{3}$ & $-1$ \\ 
	\hline
\end{tabular}\\
\renewcommand{\arraystretch}{1.0}

\newpage
\subsection{Sinusssatz}
\begin{figure}[h!]
\centering
    \includegraphics[width=0.5\textwidth]{images/sinussatz.png}
    \caption{Quelle: https://de.wikipedia.org/}
    
\end{figure}
\subsection{Cosinusssatz}
$a^2 + b^2 -2ab*cos(\gamma)= c^2$


\subsection{Standardintegrale}
\begin{minipage}{0.49\linewidth}
$\frac{d}{dx} arcsin(x) = \frac{1}{\sqrt{1-x^2}}$ \\
$\frac{d}{dx} arccos(x) = \frac{-1}{\sqrt{1-x^2}}$ \\
$\frac{d}{dx} arctan(x) =\frac {1}{x^2+1}$ \\
\end{minipage}
\begin{minipage}{0.49\linewidth}
$\frac{d}{dx} arsinh(x) =\frac{1}{\sqrt{x^2+1}}$ \\
$\frac{d}{dx} arcosh(x) =\frac{1}{\sqrt{x^2-1}}, $wenn $ x>1$ \\
$\frac{d}{dx} artanh(x) =\frac{1}{1-x^2}, $wenn $ |x|<1$ \\
\end{minipage}

\subsection{Euler Formel}
$exp(i \phi) = cos(\phi) + isin(\phi)$
\\
$exp(-i \phi) = cos(-\phi) + isin(-\phi) \Longleftrightarrow $ 
$exp(-i \phi) = cos(\phi) - isin(\phi)$ 

Daraus kann nun sin, sinh, cos und cosh in Termen von exp(x) ausgedrückt werden.

$\frac{exp(i \phi) + exp(-i \phi)}{2}  = cos(\phi)$\\
$\frac{exp(i \phi) - exp(-i \phi)}{2i}  = sin(\phi)$

Ignoriere alle i, dann folgt...

$\frac{exp(\phi) + exp(-\phi)}{2}  = cosh(\phi)$\\
$\frac{exp(\phi) - exp(-\phi)}{2}  = sinh(\phi)$




\subsection{Ableitungen, Integrale}
\subsection{Ableitungen}
$\frac{d}{dx}sin(x) = cos(x)$

$\frac{d}{dx}cos(x) = -sin(x)$

$\frac{d}{dx}tan(x) = \frac{d}{dx} \frac{sin(x)}{cos(x)} = \frac{cos(x)cos(x)- sin(x)sin(x)}{cos^2(x)} = 1 - \frac{sin^2(x)}{cos^2(x)} = 1 - tan^2(x) = \frac{1}{cos^2(x)}$

$\frac{d}{dx} \frac{1}{sin(x)} = \frac{0*sin(x) - 1*cos(x)}{sin^2(x)} = \frac{-cos(x)}{sin^2(x)}$ 

$\frac{d}{dx} \frac{1}{cos(x)} = \frac{0*cos(x) - 1*(-sin(x))}{cos^2(x)} = \frac{sin(x)}{cos^2(x)}$

$\frac{d}{dx}sin^2(x) = sin(x)*cos(x)+ cos(x)*sin(x) = 2*sin(x)*cos(x)$

$\frac{d}{dx}cos^2(x) = cos(x)*(-sin)(x)+ (-sin(x))*cos(x) = -2*sin(x)*cos(x)$ 
\newpage
\subsection{Rechenregeln}
\subsection{Additionstheoreme}
$sin^2(x)+cos^2(x) = 1$

$sin(x \pm y) = sin(x)cos(y) \pm cos(x)sin(y)$    \ \ $(\# umgekehrteAbleitungsregel)$

$cos(x \pm y) = cos(x)cos(y) \mp \sinx \siny$

$tan(x \pm y) = \frac{\tanx \pm \tany}{ 1 \mp \tanx \; \tany } = \frac{ sin(x \pm y) }{cos(x \pm y) }$

\subsection{Doppelwinkel}
$sin(2x)= 2\sinx \cosx = \frac{2 \tanx}{ 1 + tan^2(x) }$

$cos(2x)= cos^2(x) - sin^2(x) = 1 - 2sin^2(x) = 2cos^2(x) - 1 = \frac{ 1 - tan^2(x) }{ 1 + tan^2(x) }$

$tan(2x)= \frac{ 2 \tanx }{ 1 - tan^2(x) } = \frac{2}{ cot(x) - \tanx }$

$cot(2x)= \frac{ cot^2(x) - 1}{2cot(x)} = \frac{cot(x) - \tanx}{2}$ \\

Beweis mit Additionstheorem



\subsection{Produkt-zu-Summen-Formel}
$\sinx*sin(y) = \frac{1}{2}(cos(x-y)-cos(x+y))$

$\cosx*cos(y) = \frac{1}{2}(cos(x-y)+cos(x+y))$

$sin(x)*cos(y) = \frac{1}{2}(sin(x-y)+sin(x+y))$ \\



\subsection{Hyperbolische Funktionen}
$sinh(z) := \frac{e^z - e^{-z}}{2} = z + \frac{z^3}{3!} + \frac{z^5}{5!} + \frac{z^7}{7!} + \dots = \sum_{n=0}^\infty \frac{z^{2n+1}}{(2n+1)!}$

$cosh(z) := \frac{e^z + e^{-z}}{2}= 1 + \frac{z^2}{2!} + \frac{z^4}{4!} + \frac{z^6}{6!} + \dots = \sum_{n=0}^\infty \frac{z^{2n}}{(2n)!}$

$sin(z)= Im(e^{iz})=\dfrac{1}{2i} (e^{iz}-e^{-iz})$ \\

$cos(z)=\text{Re}(e^{iz})=\dfrac{1}{2}(e^{iz}+e^{-iz})$ \\

$sinh(\pm iz) = \pm i \cdot \sin(z)$\\
$cosh(\pm iz) = cos(z)$\\
$sin(iz)= i\cdot sinh(z)$\\
$cos(iz)=cosh(z)$\\

$ \sin(-z) = - \sin(z) $\\
$ \tan-(z) = -\tan(z) $\\
$ \cos(-z) = \cos(z) $ \\
$ \arctan(-z) = -\arctan(z) $

$\sin(z) = \sin(x)\cosh(y) + i\cos(x)\sinh(y)$\\
$\cos(z) = \cos(x)\cosh(y) - i\sin(x)\sinh(y)$\\
$e^z = e^x \cos(y) + i e^x \sin(y)$\\
$\sinh(z) = \cos(y)\sinh(x) + i\sin(y)\cosh(x)$\\
$\cosh(z) = \cos(y)\cosh(x) + i\sin(y)\sinh(x)$	

\begin{itemize}[leftmargin=*]
	\item $\int \sinh(ax + b) \,dx = \frac{\cosh(ax + b)}{a}$; $\int \sinh(x) \,dx
	= \cosh(x)$
	\item $\int \cosh(ax + b) \,dx = \frac{\sinh(ax + b)}{a}$; $\int \cosh(x) \,dx
	= \sinh(x)$
	\item $\int \tan(ax + b) \,dx = \frac{\log(\cosh(ax+b))}{a}$; $\int \tan(x)
	\,dx = \log(\cosh(x))$
\end{itemize}


\subsection{Additionstheoreme}
$sinh(z_1 \pm z_2) = sinh(z_1) \cdot cosh(z_2) \pm sinh(z_2) \cdot cosh(z_1)$

$cosh(z_1 \pm z_2) = cosh(z_1) \cdot cosh(z_2) \pm sinh(z_1) \cdot sinh(z_2)$

$tanh(z_1 \pm z_2) = \frac{tanh(z_1) \pm tanh(z_2)}{1 \pm tanh(z_1) \cdot tanh(z_2)}$


\subsubsection{Zusammenhänge}
$cosh^2(z) - sinh^2(z) = 1$ \hfill $cosh(z) + sinh(z) = e^z$ \hfill $cosh(z) - sinh(z) = e^{-z}$

\subsection{Ableitungen}
$\frac{d}{dz}sinh(z) = cosh(z)$ \hfill $\frac{d}{dz}cosh(z) = sinh(z)$ \hfill $\frac{d}{dz}tanh(z) = 1 -tanh^2(z) = \frac{1}{cosh^2(x)}$

\section{Plots Trigonometrischer Funktionen}
%Alle Plots in diesem Kapitel von www.wikipedia.org!

\begin{figure}[!htb]
	\centering
	\begin{minipage}{.5\textwidth}
		\centering
		\includegraphics[width=1\linewidth]{images/tan.png}
	\end{minipage}%
	\begin{minipage}{0.5\textwidth}
		\centering
		\includegraphics{images/asinacos.png}
	\end{minipage}
\end{figure}


\begin{figure}[H] 
	\centering
	{\includegraphics[width=0.55\textwidth]{images/arctan.png}}
\end{figure}



\begin{figure}[!htb]
	\centering
	\begin{minipage}{.5\textwidth}
		\centering
		\includegraphics[width=1\linewidth]{images/sinhcoshtanh.png}
	\end{minipage}%
	\begin{minipage}{.5\textwidth}
		\centering
		\includegraphics[width=1\linewidth]{images/arsinharcosh.png}
	\end{minipage}
\end{figure}


\input{src/induktion.tex}
\section{Mengen}

\subsection{Definitionen}

\begin{description}[labelindent=16pt,style=multiline,leftmargin=6cm, noitemsep]
	\item[Obere/Untere Schranke:] $\exists b \in \mathbb{R}\ \forall a\in A:\ a \leq b$, $\exists c \in \mathbb{R}\ \forall a\in A:\ a \geq c$
	\item[Supremum:] kleinste obere Schranke $\sup A$
	\item[Infimum:] gr{\"o}sste untere Schranke $\inf A$
	\item[Maximum/Minimum:] $\sup A \in A$, $\inf A \in A$
	\item[kompakt:] abgeschlossen und beschr{\"a}nkt
	\item[abgeschlossen:] z.B. $[0,1]$
\end{description}

\subsubsection{Vorgehen zur Bestimmung von Maximum/Minimum}

\begin{enumerate}[noitemsep]
	\item Zeigen, dass $f(x)$ stetig ist
	\item Zeigen, dass Definitionsmenge kompakt ist
	\item Nach \textbf{Satz von Weierstrass} wird Maximum/Minimum angenommen
	\item Maximum/Minimum bestimmen
\end{enumerate}

\subsection{Identit{\"a}ten}

\begin{equation*}
\begin{split}
A + B & := \{a + b | a \in A, b \in B\} \\
\sup(A+B) = \sup A + \sup B,\ & \inf(A+B) = \inf A + \inf B \\
\sup(A \cup B) = \max\{\sup A, \sup B\},\ & \inf(A \cup B) = \min\{\inf A, \inf B\}
\end{split}
\end{equation*}


\end{document}


