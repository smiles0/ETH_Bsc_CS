
\section{Differenzialrechnung}

Eine stetige Funktion ist differenzierbar, falls der Grenzwert $f'(x_0)$ existiert:

\begin{equation*}
f'(x_0) := \lim_{x\to x_0}\frac{f(x) - f(x_0)}{x-x_0}
\end{equation*}
\subsubsection*{Tangente}
Sei $f: \Omega \to \mathbb{R}$ an der Stelle $x_0 \in \Omega$ diffbar. Dann ist die Tangente im Punkt $x_0$
\[ t(x; x_0)=f(x_0)+f'(x_0)(x-x_0) \]

\subsection{Umkehrsatz}

\begin{equation*}
(f^{-1})'(y) = \frac{1}{f'(f^{-1}(y))}
\end{equation*}

\subsection{Mittelwertsatz}

\begin{equation*}
f'(c) = \frac{f(b) - f(a)}{b - a}
\end{equation*}

\subsection{Taylorpolynom}

Das Taylorpolynom $m$-ter Ordnung von $f(x)$ an der Stelle $x=a$
\begin{equation*}
P^a_m(x) := f(a) + f'(a)(x-a) + \frac{1}{2}f''(a)(x-a)^2 + ... + \frac{1}{m!} f^{(m)}(a)(x-a)^m
\end{equation*}

mit dem Fehlerterm $R^a_m(x)$, wobei $\xi$ zwischen $a$ und $b$ liegt:
\begin{equation*}
R^a_m(x) = \frac{f^{(m+1)}(\xi)}{(m+1)!}(x+a)^{m+1},\ \text{wobei}\ f(x) = P^a_m(x) + R^a_m(x)
\end{equation*}
