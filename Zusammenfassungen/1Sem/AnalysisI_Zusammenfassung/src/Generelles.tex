\section{Mengen}

\subsection{Definitionen}

\begin{description}[labelindent=16pt,style=multiline,leftmargin=6cm, noitemsep]
	\item[Obere/Untere Schranke:] $\exists b \in \mathbb{R}\ \forall a\in A:\ a \leq b$, $\exists c \in \mathbb{R}\ \forall a\in A:\ a \geq c$
	\item[Supremum:] kleinste obere Schranke $\sup A$
	\item[Infimum:] gr{\"o}sste untere Schranke $\inf A$
	\item[Maximum/Minimum:] $\sup A \in A$, $\inf A \in A$
	\item[kompakt:] abgeschlossen und beschr{\"a}nkt
	\item[abgeschlossen:] z.B. $[0,1]$
\end{description}

\subsubsection{Vorgehen zur Bestimmung von Maximum/Minimum}

\begin{enumerate}[noitemsep]
	\item Zeigen, dass $f(x)$ stetig ist
	\item Zeigen, dass Definitionsmenge kompakt ist
	\item Nach \textbf{Satz von Weierstrass} wird Maximum/Minimum angenommen
	\item Maximum/Minimum bestimmen
\end{enumerate}

\subsection{Identit{\"a}ten}

\begin{equation*}
\begin{split}
A + B & := \{a + b | a \in A, b \in B\} \\
\sup(A+B) = \sup A + \sup B,\ & \inf(A+B) = \inf A + \inf B \\
\sup(A \cup B) = \max\{\sup A, \sup B\},\ & \inf(A \cup B) = \min\{\inf A, \inf B\}
\end{split}
\end{equation*}