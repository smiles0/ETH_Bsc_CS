\section{Vollständige Induktion}\index{Induktion}
Grundlägende Struktur um die Aussage $A(n)$ zu beweisen:
\begin{enumerate}
	\item \textbf{Induktionsanfang/Verankerung:} Die Aussage wird für $n = A$ bewiesen.
	$A$ ist dabei meistens der erste Wert für die gegebene Eingabemenge.
	Der Beweis wird meist durch direktes ausrechnen gemacht.
	\item \textbf{Annahme/Induktionsvoraussetzung:} Hier schreibt man,
	dass man davon ausgeht die Aussage sei gültig (damit man sie im nächsten Schritt)
	einsetzen kann. Man kopiert also im Grunde, was man zu beweisen hat mit einigen Zierwörter.
	\item \textbf{Induktionsschritt:} Für jedes $n \geq A$ wird unter Benutzung der Aussage $A(n)$
	die Aussage $A(n+1)$ bewiesen. Dazu wird die Induktionsannahme verwendet.
\end{enumerate}

\subsection{Beispiel}
		Es ist zu beweisen, dass für jedes $n \in \N$ folgendes gilt: $1 + 2 + 3 + \ldots + n = \frac{n(n + 1)}{2}$		
		
		Lemma: $\forall \in \N. 1 + 2 + 3 + \ldots + n = \frac{n(n + 1)}{2}$ \\
		
	Beweis: \\
		  Sei $P(n) \equiv 1 + 2 + 3 + \ldots + n = \frac{n(n + 1)}{2}$. \\
		  Wir zeigen $\forall \in \N. P(n)$ mit vollständiger Induktion.\\
		
		Induktionsanfang: Zeige $P(0)$. \\
		 $$0 =  \frac{0(0 + 1)}{2}$$
		
		  Induktionsschritt: \\
		    Sei $n \in N$ beliebig und nehmen wir $P(n)$ an (Induktionsvoraussetzung).  \\
		    Zeige $P(n+1)$ (Induktionsbehauptung).\\
		  \begin{align}
		      1 + 2 + 3 + \ldots + n + (n + 1) =\\
		      = \frac{n(n + 1)}{2} + (n+1)       & \text{ — P(n) Induktionsvor. }\\
		      = \frac{n(n + 1)}{2} + \frac{2(n + 1)}{2}  & \text{—arith}\\
		      = \frac{(n+1)*((n+1)+1)}{2}        & \text{—arith}
		\end{align}
		qed.