\section{Komplexe Zahlen und Funktionen}
\subsection{Komplexe Zahlen - Grundlagen}

\begin{itemize}[noitemsep,topsep=0pt]
	
\item	\(i  = \sqrt{-1}\)
\item	\(z  = x + iy = r(\cos(\varphi) + i\sin(\varphi)) = re^{i\varphi} \)
\item	\(r  = |z| = \sqrt{x^2 + y^2} \)
\item	\(\arg(z)  = \varphi  = \arctan(\frac{y}{x}) \quad \text{(je nach Quadrant)}  \)
\item	\(x  = r\cos(\varphi) \)
\item	\(y  = r\sin(\varphi) \)
\item	\(zw  = (re^{i\varphi})\cdot(se^{i\psi}) = rse^{i(\varphi + \psi)} \)
\item	\(e^{i(\frac{\pi}{2} + 2\pi k)}  = i,\ e^{i\pi} = 1, \ e^{-i\pi} = -1\)

\end{itemize}

\subsection{Rechenregeln}
\begin{minipage}[h]{0.45\linewidth}

\begin{itemize}[noitemsep,topsep=0pt]
\item $ x = \text{Re } z = \frac{z + \overline{z}}{2} $
\item $ y = \text{Im } z = \frac{z - \overline{z}}{2i}$
\item $ z \in \mathbb{R} \Longleftrightarrow z = \overline{z} $
\item $ \overline{\overline{z}} = z $
\item $ \overline{ \left( \frac{1}{z} \right)} = \frac{1}{(\overline{z})}$
\item $ \overline{z_1 + z_2} = \overline{z_1} + \overline{z_2}$
\item $ \overline{z_1 \cdot z_2} = \overline{z_1} \cdot \overline{z_2} $	
\item	\((a,b) \cdot (c, d)  = (ac-bd, ad+bc) \)
\item	\(|z|^2  = z\overline{z} \)
\item	\(|zw|^2  = (zw) \cdot \overline{(zw)} = |z|^2|w|^2\)
\item	\(i^2 = (-i)^2 = -1\) und \\ \( \dfrac{1}{i} = -\dfrac{-1}{i} = -\dfrac{i^2}{i} = -i \)
\end{itemize}
\end{minipage}
\hfill
\begin{minipage}[h]{0.45\linewidth}
\begin{itemize}[noitemsep,topsep=0pt]
\item	\(z=x+iy\) mit \(z \in \mathbb{C}\)
\item	\(z + z' = (x + x') + i(y + y')\)
\item	\(z \cdot z' = xx' - yy' + i(x'y + y'x)\)
\item	\(\alpha z = \alpha x + i \alpha y\)
\item	\(\dfrac{1}{z} = \dfrac{\overline{z}}{z\overline{z}} = \dfrac{x- iy}{x^2 + y^2} \)
\item	\(\overline{z} = x - iy = \overline{r\cdot e^{i\varphi}} = r\cdot e^{-i\varphi}\)
\item	\(e^z=e^{x+iy}=e^x(\cos(y)+i\sin(y))\)
\item	\(z^n = (r\cdot e^{i\varphi})^n = r^n\cdot e^{i\varphi n}\)	\\				

\end{itemize}


\end{minipage}


\subsection{Betrag}

\begin{itemize}[noitemsep,topsep=0pt]
	\item \(\vert z \vert = \sqrt{z \cdot \overline{z}} = \sqrt{x^2 + y^2}\) und somit auch \(\vert z \vert^2 = z\cdot \overline{z} = x^2+y^2\)
	\item $ \vert z \cdot z' \vert = \vert z \vert \cdot \vert z' \vert $ (im komplexen!)
	\item $ z \in \mathbb{R} \implies \vert z \vert_\mathbb{C} = \vert z \vert_\mathbb{R} $
	\item $ \vert \text{Re } z \vert \leqslant \vert z \vert $, $ \vert \text{Im } z \vert \leqslant \vert z \vert $
	\item $ \vert z + z' \vert \leqslant \vert z \vert + \vert z' \vert $ (Dreiecksungleichung)
	\item $ \vert e^z \vert = e^{\text{Re }z} $
	\item $ z^2 - \overline{z}^2 = 4i \text{Re }(z)\text{Im }(z) $
\end{itemize}


Der K"orper $\mathbb{C}$ ist nicht geordnet und eine \textbf{Ungleichung} wie $z_1<z_2$ \textbf{macht keinen Sinn}!


\subsection{Norm}
$\vert \vert f(t) \vert \vert^2 = \frac{1}{2\pi} \int_{-\pi}^{\pi} \vert f(t) \vert^2 dt$

\subsection{Mitternacht}
\eqbox{ az^2 + bz + c = 0 \Leftrightarrow z_{1,2} = \dfrac{-b \pm \sqrt{b^2-4ac}}{2a}}

\subsection{Polardarstellung}
\textbf{Form}\\
\eqbox{z = r e^{i \varphi} = r(\cos\varphi + i \sin\varphi)} mit \(r\in \mathbb{R}^+ (r\geqslant 0)\)\\
\textbf{kartesisch\(\to\)polar}\\
\(r = \vert z \vert =\sqrt{x^2+y^2}\)\\
$ \arg(z) = \arg(x,y) = \{ \varphi + 2k\pi \vert k \in \mathbb{Z} \} \implies\) \eqbox{\varphi \in \arg z} (Menge)\\
Innerhalb $[-\pi, \pi]$ l"asst sich $\varphi$ so berechnen:\\
\(\varphi = \begin{cases}
\arctan \frac{y}{x}		&	\text{f"ur } x>0\\
\arctan \frac{y}{x}+\pi		&	\text{f"ur } x<0, y \geqslant 0\\
\arctan \frac{y}{x}-\pi		&	\text{f"ur } x<0, y<0\\
+\pi / 2				&	\text{f"ur } x=0, y>0\\
-\pi / 2				&	\text{f"ur } x=0, y<0\\
\text{\emph{undef.}}		&	\text{f"ur } x=0, y=0
\end{cases}\)\\
\textbf{polar\(\to\)kartesisch}\\
$ x = r \cos \varphi $\\
$ y = r \sin \varphi $\\
\textbf{komplexe Multiplikation}\\
\(z_1 \cdot z_2 = r_1 \cdot r_2 e^{i(\varphi_1 + \varphi_2)}\)
\(z^n = r^n \cdot e^{in\varphi}\)\\
\textbf{n-te Wurzel} \(\implies\) genau $n$ L"osungen!\\
\eqbox{\sqrt[n]{z} = w_k = \vert z \vert ^{\frac{1}{n}} e^{i \left( \frac{\varphi}{n} + \frac{2k\pi}{n} \right)} } mit $ k = 0,1,\dots, n-1$

\textbf{Hauptwert des Arguments} (eindeutig!)\\
$ -\pi < \varphi < \pi $, mit \eqbox{\varphi = \text{Arg}(z)} \(\implies\) \eqbox{\text{Arg } \overline{z} = - \text{Arg } z}\\
$z$ liegt auf der positiven reellen Achse: $\GDW \text{Arg } z = 0 $\\
$z$ auf negativen reellen Achse \(\GDW\) Arg-Funktion kann \(z\) \underline{nicht} abbliden
%\newline

\begin{minipage}[t]{0.48\linewidth}
	
\subsection{Gamma-Funktion}
$\Gamma (\alpha):= \int_{0}^{\infty} x^{\alpha - 1} e{-x} dx$ \\

Was gilt:
\begin{itemize}
\item $\Gamma (\alpha + 1) = \alpha \cdot \Gamma (\alpha)$
\item $\Gamma (n) = (n-1)!, \forall n \in \mathbb{N}$
\item $\Gamma (\frac{1}{2}) = \sqrt{\pi} $
\item $\Gamma (\alpha) \cdot \Gamma (1-\alpha) = \frac{\pi}{sin(\pi \cdot \alpha)}, \alpha \in [0,1]$
\end{itemize}
\end{minipage}
\hfill
\begin{minipage}[t]{0.45\linewidth}
\subsection{Dirac-Delta Funktion}

$\delta(t) = \begin{cases}
\infty,  \hspace{1cm} falls \hspace{0.5cm}t = 0\\
0,  \hspace{1cm} falls \hspace{0.5cm} t \neq 0
\end{cases}$

\end{minipage}