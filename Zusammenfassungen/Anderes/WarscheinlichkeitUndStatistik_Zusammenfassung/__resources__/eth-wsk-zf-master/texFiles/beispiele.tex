\section{Beispiele}

\subsection{t-Test}
Ein Waschmittelhersteller bringt 5kg-Packungen in den Umlauf. Die Konsumentenschutzorganisation kauft 25 Packungen. Es ergibt sich ein Mittel von $\overline{X}=4.9kg$ und eine empirische Stichprobenvarianz $S^2 = 0.1kg^2$. Die einzelnen Gewichte seine durch unabhängige $N(\mu,\sigma^2)$ Zufallsvariablen beschrieben.

\begin{enumerate}
\item Wie lauten die Hypothesen $H_0$ und $H_A$?
\item Wie ist $(\overline{X}-5) / (S / 5))$ verteilt unter $H_0$?
\item Führen Sie den t-Test auf dem 5%-Niveau durch. Wie gross ist der P-Wert? Wird die Nullhypothese verworfen?
\item Berechnen Sie das 95\%-Vertrauensintervall für den in 3. beschriebenen Test.
\end{enumerate}

\vspace{10pt}

\textbf{Lösung:}
\begin{enumerate}
\item $H_0: \mu = 5, H_1: \mu < 5$
\item Verteilung: $\sim t_{5^2-1} = t_{24}$
\item $T_1 = \frac{\overline{X}-\mu_0}{S/\sqrt{n}} = -1.581$. Wir ermitteln das t-Quantil für $n-1 = 24$ Freiheitsgrade und $t^{-1}(\alpha) = t^{-1}(0.05) = t^{-1}(0.95) = 1.711$. Somit erhalten wir eine kritische Grenze von $c = -1.711$. Weil $T_1 = -1.581 > c$ ist, wird die Nullhypothese nicht verworfen.
\item ???
\end{enumerate}

\subsection{P-Wert berechnen}
Für einen t-Test oder einen z-Test kann der P-Wert wie folgt berechnet werden:

\vspace{10pt}

\textbf{P-Wert bei zweiseitigem Test:} \\
Beispiel mit $c=3.43$ als Grenze des Verwerfungsbereichs:
\[
\begin{array}{rcl}
	P_{H_0}[|T|>3.43] & = & 2\cdot P_{H_0}[T>3.43] \\
	& = & 2\cdot (1-P_{H_0}[T\leq 3.43] \\
	& = & 2\cdot (1-\Phi(3.43)) \\
	& = & 2\cdot (1-0.999698) \\	
	& \approx & 0.0006
\end{array}
\]

\vspace{10pt}

\textbf{P-Wert bei einseitigem Test (rechts):} \\
Beispiel mit $c=3.43$ als Grenze des Verwerfungsbereichs:
\[
\begin{array}{rcl}
	P_{H_0}[T>3.43] & = & (1-P_{H_0}[T\leq 3.43] \\
	& = & (1-\Phi(3.43)) \\
	& = & (1-0.999698) \\	
	& \approx & 0.0003
\end{array}
\]

\vspace{10pt}

\textbf{P-Wert bei einseitigem Test (links):} \\
Beispiel mit $c=3.43$ als Grenze des Verwerfungsbereichs:
\[
\begin{array}{rcl}
	P_{H_0}[T<3.43] & = & \Phi(3.43) \\
	& \approx & 0.999698) \\
\end{array}
\]

\subsection{Likelihood-Quotienten-Test (Neyman-Pearson-Lemma}
Du erhältst den Auftrag, die Anzahl Ausfälle eines Systems zu überprüfen. Nach Angaben des Herstellers sollen erwartungsgemäss 0.5 Ausfälle/h eintreten. Gehe davon aus, dass die Anzahl Ausfälle poisson-verteilt mit unbekanntem Parameter $\lambda$ ist und dass die einzelnen Ausfälle unabhängig voneinander sind. Die Analyse nach 6 Betriebsstunden hat nun folgendes Ergebnis gebracht:

\begin{center}
  \begin{tabular}{|l|c|c|c|c|c|c|}
    \hline
    \textbf{Betriebsstunde i} & 1 & 2 & 3 & 4 & 5 & 6 \\ \hline
    \textbf{Anz. Ausfälle $X_i$} & 1 & 0 & 1 & 1 & 2 & 1 \\
    \hline
  \end{tabular}
\end{center}

Aufgrund der hohen Anzahl Ausfälle haben wir den Verdacht, dass $\lambda$ grösser als die vom Hersteller angegebenen Anzahl Ausfälle ist. Prüfe anhand eines einseitigen Tests auf dem Niveau 2.5\%, ob tatsächlich $\lambda=0.5$ Ausfälle/h angenommen werden kann. Gib
\begin{enumerate}
\item das Modell,
\item die Nullhypothese,
\item die Alternativhypothese,
\item die Teststatistik,
\item die Verteilung der Teststatistik unter der Nullhypothese,
\item den Verwerfungsbereich,
\item den beobachteten Wert der Teststatistik, sowie
\item den Testentscheid
\end{enumerate}
an.

\vspace{10pt}

\textbf{Lösung:}
\begin{enumerate}
\item \textbf{Modell:} Unter $P_{\lambda}$ sind die $X_i,\ i.i.d.\sim Pois(\lambda),\ i=1,...,6,\ \lambda$ unbekannt.
\item \textbf{Nullhypothese:} $H_0:\ \lambda =\lambda_0=0.5$
\item \textbf{Alternativhypothese:} $H_A:\ \lambda =\lambda_A >\lambda_0$
\item \textbf{Teststatistik:} $T=\sum\limits_{i=1}^{6}X_i$, denn
\[
R(x_1,...,x_6;\lambda_0,\lambda_A) =
\frac{
	L(x_1,x_2,...,x_6;\lambda_0)
}{
	L(x_1,x_2,...,x_6;\lambda_A)
} =
\]
\[
\frac{
	e^{-6\lambda_0}\prod\limits_{i=1}^{6}\frac{\lambda_{0}^{x_i}}{x_i!}
}{
	e^{-6\lambda_A}\prod\limits_{i=1}^{6}\frac{\lambda_{A}^{x_i}}{x_i!}
} =
e^{-6(\lambda_0-\lambda_A)}\left(\frac{\lambda_0}{\lambda_A}\right)^{\sum\limits_{i=1}^{6}x_i}
\]
Da $\lambda_0<\lambda_A$ wird $R(x_1,...,x_6;\lambda_0,\lambda_A)$ klein, genau dann, wenn $\sum\limits_{i=1}^{6}x_i$ gross ist. Statt des komplizierten Quotienten wählen wir als Teststatistik also
\[
T=\sum\limits_{i=1}^{6}X_i
\]
\item \textbf{Verteilung der Teststatistik unter $H_0$:} $T\sim Pois(6\lambda_0)=Pois(3)$
\item \textbf{Verwerfungsbereich:} Der kritische Bereich "'Quotient klein"' hat die äquivalente Form "'Summe gross"', also ist der Verwerfungsbereich von der Form $K=[k,\infty)$. Um das Signifikanzniveau einzuhalten, muss gelten $P_{\lambda_0}[T\geq k]\leq 2.5\% \Leftrightarrow P_{\lambda_0}[T<k]\geq 97.5\%$:

\begin{center}
  \begin{tabular}{c|c|c}
    k & $P_{\lambda_0}[T=k]$ & $P_{\lambda_0}[T\leq k]$ \\ \hline
    0 & 0.050 & 0.050 \\
    1 & 0.149 & 0.199 \\
    2 & 0.224 & 0.423 \\
    3 & 0.224 & 0.647 \\
    4 & 0.168 & 0.815 \\
    5 & 0.101 & 0.916 \\
    6 & 0.050 & 0.966 \\
    7 & 0.022 & 0.988 \\
  \end{tabular}
\end{center}

Deshalb haben wir als Verwerfungsbereich $=[8,\infty)$.
\item \textbf{Beobachteter Wert der Teststatistik:} $t=6$
\item \textbf{Testentscheid:} Da $6$ nicht im Verwerfungsbereich liegt, wird die Nullhypothese \textbf{nicht} verworfen.


\end{enumerate}





\subsection{Erwartungstreuer Schätzer}
Wir prüfen, ob der Schätzer $T = \frac{1}{n}\sum\limits_{i=1}^{n}X_i$ erwartungstreu ist.

\[
E[T] = E\left[\frac{1}{n}\sum\limits_{i=1}^{n}X_i\right] = \frac{1}{n}\sum\limits_{i=0}^{n}E[X_i] = \frac{1}{n} \cdot n \cdot \mu = \mu
\]

Der Schätzer $T$ ist erwartungstreu.

\subsection{Verteilung der Summe zweier normalerverteilter ZV}
Seien $X\sim N(\mu_1,\sigma_1^2)$ und $Y\sim N(\mu_2,\sigma_2^2)$. Wie ist $Z=1+aX+bY$ verteilt?

\vspace{10pt}

\textbf{Lösung:} \\
\[
\begin{array}{rcl}
	Z & \sim & N(1+aE[X]+bE[Y],a^2Var[X]+b^2Var[Y]) \\
	& \sim & N(1+a\mu_1+b\mu_2,a^2\sigma_1^2+b^2\sigma_2^2) \\
\end{array}
\]