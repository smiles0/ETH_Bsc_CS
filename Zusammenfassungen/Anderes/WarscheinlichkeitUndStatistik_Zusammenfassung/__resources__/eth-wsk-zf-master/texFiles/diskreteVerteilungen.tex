\section{Diskrete Verteilungen}

\subsection{Bernoulli-Verteilung}
Verteilung eines Experiments mit zwei Ausgängen (Erfolg, Misserfolg) mit Erfolgsparameter $p$.

\vspace{10pt}

Notation:
\[
X \sim Be(p)
\]

Gewichtsfunktion:
\[
f(x)=P[X=x]=p^x\cdot (1-p)^{1-x}
\]

Erwartungswert:
\[
E[X]=p
\]

Varianz:
\[
Var[X]=p(1-p)
\]

\textbf{Beispiel (Münzwurf):} Ein fairer Münzwurf ist bernoulliverteilt mit Parameter $p=\frac{1}{2}$. Für einen Parameter $p\neq\frac{1}{2}$ wäre der Münzwurf unfair.

\subsection{Binomial-Verteilung}
Beschreibt die Anzahl der Erfolge in einer Serie von \emph{gleichartigen} und \emph{unabhängigen} Versuchen, die jeweils genau zwei mögliche Ergebnisse haben (''Erfolg'' oder ''Misserfolg''). $n$ ist die Anzahl der Versuche bzw. Wiederholungen, $p$ ist die Wahrscheinlichkeit für einen "Erfolg".

\vspace{10pt}

Notation:
\[
X \sim Bin(n,p)
\]

Gewichtsfunktion:
\[
f(x)=P[X=x]=
\left(
	\begin{array}{c}
		n \\
		x
	\end{array}
\right)
p^x(1-p)^{n-x}
\]

Verteilungsfunktion:
\[
F(x)=P[X \leq x]=\sum \limits_{k=0}^x 
\left(
	\begin{array}{c}
		n \\
		k
	\end{array}
\right)
p^k(1-p)^{n-k}
\]

Erwartungswert:
\[
E[X]=np
\]

Varianz:
\[
Var[X]=np(1-p)
\]

\textbf{Bemerkungen:}
\begin{itemize}
\item Die Binomialverteilung $Bin(n,p)$ darf für grosse $n$ und kleine $p$ näherungsweise durch die Poisson-Verteilung mit Parameter $\lambda = np$ ersetzt werden (Faustregel: $np<10$ und $n>1500p$).
\end{itemize}

\textbf{Beispiel:} Geburtstagsproblem, Anzahl der Köpfe beim 10-maligen Münzwurf.

\subsection{Geometrische Verteilung}
Wartezeit auf ersten Erfolg bei einer Folge von 0-1-Experimenten mit Erfolgsparameter $p$. $X$ ist die Nummer des ersten erfolgreichen Experiments.

\vspace{10pt}

Notation:
\[
X \sim Geom(p)
\]

Gewichtsfunktion:
\[
f(x)=P[X=x]=p(1-p)^{x-1}
\]

Verteilungsfunktion:
\[
F(x)=P[X \leq x]=1-(1-p)^{x}
\]

Erwartungswert:
\[
E[X]=\frac{1}{p}
\]

Varianz:
\[
Var[X]=\frac{1-p}{p^2}
\]

\textbf{Beispiel:} Wartezeit auf Kopf bei wiederholtem Münzwurf

\subsection{Poisson-Verteilung}
Ereignet sich in einem Intervall (z.B. in einer gewissen Zeit, auf einer gewissen Fläche, in einem gewissen Volumen, usw.) ein völlig zufällig auftretendes Ereignis im Schnitt $\mu$ mal (Erwartungswert), dann ist die Zufallsgrösse, welche die Häufigkeit des Ereignisses in diesem Intervall angibt, poissonverteilt mit Parameter $\lambda=\mu$.

\vspace{10pt}

Notation:
\[
X \sim P(\lambda), X \sim Pois(\lambda)
\]

Gewichtsfunktion:
\[
f(x)=P[X=x]=e^{-\lambda}\frac{\lambda^{x}}{x!}
\]

Verteilungsfunktion:
\[
F(x)=P[X \leq x]=e^{-\lambda}\sum \limits_{k=0}^x \frac{\lambda^{k}}{k!}
\]

Erwartungswert:
\[
E[X]=\lambda
\]

Varianz:
\[
Var[X]=\lambda
\]

\textbf{Bemerkungen}
\begin{itemize}
\item Falls $Z=X+Y$ und $X\sim Pois(\lambda)$, $Y\sim Pois(\mu)$ dann gilt $Z\sim Pois(\lambda+\mu)$.
\end{itemize}

\textbf{Beispiel:} Anzahl eingehender Druckaufträge in $n=3600$ Sekunden mit einer Jobwahrscheinlichkeit von $p=1\%$ pro Sekunde, wenn pro Sekunde maximal ein Job eintreffen kann.

\subsection{Negativbinomiale Verteilung}
Wartezeit auf den $r$-ten Erfolg bei einer Folge von unabhängigen 0-1-Experimenten mit Erfolgsparameter $p$.

Notation:
\[
X \sim NB(r,p)
\]

Gewichtsfunktion:
\[
f(x)=P[X=x]=\left(
	\begin{array}{c}
		x-1 \\
		r-1
	\end{array}
\right)p^{r}(1-p)^{x-r}
\]

Verteilungsfunktion:
\[
F(x)=P[X \leq x]=\sum \limits_{i=r}^x
\left(
	\begin{array}{c}
		i-1 \\
		r-1
	\end{array}
\right)p^{r}(1-p)^{i-r}
\]

Erwartungswert:
\[
E[X]=\frac{r}{p}
\]

Varianz:
\[
Var[X]=\frac{r(1-p)}{p^2}
\]

\textbf{Beispiel:} Wartezeit auf dritten Kopf beim wiederholten Münzwurf

\subsection{Hypergeometrische Verteilung}
Urnenmodell für Ziehen ohne Zurücklegen. In einer Urne befinden sich $n$ Gegenstände. Davon sind $r$ Gegenstände vom Typ A und $n-r$ Gegenstände vom Typ B. Es werden $m$ Gegenstände ohne Zurücklegen gezogen. $X$ beschreibt die Wahrscheinlichkeitsverteilung für die Anzahl $k$ der Gegenstände vom Typ A in der Stichprobe.

\vspace{10pt}

Notation:
\[
X \sim HypGeom(n,m,r)
\]

Gewichtsfunktion:
\[
f(x)=P[X=x]=\frac{
\left(
	\begin{array}{c}
		r \\ x
	\end{array}
\right)
\left(
	\begin{array}{c}
		n-r \\ m-x
	\end{array}
\right)
}{
\left(
	\begin{array}{c}
		n \\ m
	\end{array}
\right)
}
\]

Verteilungsfunktion:
\[
F(x)=P[X \leq x]=
\sum \limits_{k=max(0,m-n)}^x \frac{
\left(
	\begin{array}{c}
		r \\ k
	\end{array}
\right)
\left(
	\begin{array}{c}
		n \\ m-k
	\end{array}
\right)
}{
\left(
	\begin{array}{c}
		r+n \\ m
	\end{array}
\right)
}
\]

Erwartungswert:
\[
E[X]=\frac{mr}{n}
\]

Varianz:
\[
Var[X]=\frac{mr}{n^2(n-1)}(n-r)(n-m)
\]

\textbf{Beispiel (Lotto):} Anzahl Zahlen $n=45$, richtige Zahlen $r=6$, meine Zahlen $m=6$. Die Wahrscheinlichkeit für $4$ Richtige ist
\[
f(4)=\frac{
	\left(
	\begin{array}{c}
		6 \\
		4 \\
	\end{array}
	\right)
	\left(
	\begin{array}{c}
		39 \\
		2 \\
	\end{array}
	\right)
}{
	\left(
	\begin{array}{c}
		35 \\
		6 \\
	\end{array}
	\right)
}\approx 0.00136
\]