\section{Statistik}

\subsection{Schätzer}
Schätzer sind Funktionen von Zufallsvariablen und somit selbst wieder Zufallsvariablen. Sie verfügen deshalb über einen Erwartungswert und eine Varianz.

\subsubsection{Erwartungstreuer Schätzer}
Ein Schätzer $T$ ist \emph{erwartungstreu} wenn der Erwartungswert des Schätzers gleich dem zu schätzenden Parameter $\vartheta$ ist:
\[
E_\vartheta [T] = \vartheta
\]

\subsubsection{Konsistenter Schätzer}
Ein Folge von Schätzern $T^{(n)}$ ist \emph{konsistent} wenn diese mit zunehmendem Stichprobenumfang $n$ gegen den gesuchten Parameter $\vartheta$ konvergiert:
\[
\lim_{n\to\infty}P_{\vartheta}\left[|T^{(n)}-\vartheta| > \epsilon\right] = 0\text{ (für jedes $\epsilon > 0$)}
\]

\subsubsection{Momenten-Methode}
Sei $X_1,X_2,...,X_n$ eine Stichprobe vom Umfang $n$ mit gegebener Verteilung $t$. Die Parameter von $t$ seien unbekannt. Mit der Momenten-Methode können diese geschätzt werden. Der Momentenschätzer ist i.d.R. nicht erwartungstreu.

\vspace{10pt}

\textbf{Vorgehen:}
\begin{enumerate}
\item 1. Moment berechnen:
\[
\overline{x} = \frac{1}{n}\sum\limits_{i=1}^{n}x_i
\]
\item 2. Moment berechnen:
\[
s^2 = \frac{1}{n}\sum\limits_{i=1}^{n}\left(x_i - \overline{x}\right)^2=\left( \frac{1}{n}\sum\limits_{i=1}^{n}x_i^2\right)-\overline{x}^2
\]
\item Nimm die Formel für den Erwartungswert der geg. Verteilung $t$ und setze sie gleich $\overline{x}$
\item Nimm die Formel für die Varianz der geg. Verteilung $t$ und setze sie gleich $s^2$
\item Löse das Gleichungssystem auf. Du erhälst die gesuchten Parameter.
\end{enumerate}

\subsubsection{Maximum-Likelihood-Methode}
Methode zur systematischen Gewinnung von Schätzfunktionen.

\vspace{10pt}

\textbf{Vorgehen:}
\begin{enumerate}
\item Likelihood-Funktion aufstellen:
\[
L(x_1,...,x_n,\vartheta)= \prod \limits_{i=1}^{n} f(x_i,\vartheta)
\]
\item Log-Likelihood-Funktion aufstellen (logarithmieren von $L$):
\[
\log L(x_1,...,x_n,\vartheta)
\]
\item $\log L$ mit Hilfe elementarer Logarithmenregeln möglichst vereinfachen
\item $\log L$ nach jedem unbekannten Parameter partiell ableiten
\item Partielle Ableitungen = 0 setzen
\item Gleichungssystem nach Parametern auflösen
\end{enumerate}

\subsection{Tests}

\subsubsection{Wichtige Begriffe}

\begin{itemize}
\item \textbf{Einseitiger Test:} Einen einseitigen Test führt man durch, wenn man wissen will, ob sich ein Wert vergrössert oder verkleinert hat.
\item \textbf{Zweiseitiger Test:} Einen zweiseitigen Test führt man durch, wenn man lediglich wissen will, ob sich ein Wert verändert hat.
\item \textbf{Nullhypothese $H_0$:} Die zu prüfende Annahme.
\item \textbf{Alternativhypothese $H_A$}: Alternative Annahme, falls $H_0$ verworfen werden muss.
\item \textbf{Signifikanzniveau $\alpha$}: Ist die Wahrscheinlichkeit, dass die Nullhypothese verworfen werden muss (auch Wahrscheinlichkeit für Fehler 1. Art).
\item \textbf{Teststatistik T:} Test- oder Prüfwert.
\item \textbf{P-Wert:} Das kleinste Niveau, auf dem der Test die Nullhypothese noch verwirft.
\item \textbf{Fehler 1. Art:} $H_0$ wird verworfen, obwohl sie richtig wäre.
\item \textbf{Fehler 2. Art:} $H_0$ wird beibehalten, obwohl $H_A$ stimmt.
\end{itemize}

\subsubsection{Allgemeines Vorgehen}
\begin{enumerate}
\item Wahl des Modells
\item Formulieren der Nullhypothese $H_0$
\item Formulieren der Alternativhypothese $H_A$
\item Teststatistik $T$ aufstellen
\item Verteilung der Teststatistik unter der Nullhypothese
\item Bestimmen des Verwerfungsbereichs
\item Konkreter Wert für Teststatistik $T$ berechnen
\item Testentscheidung: $H_0$ beibehalten oder verwerfen?
\end{enumerate}

\subsubsection{z-Test}
Der z-Test ist ein Test für den Erwartungswert bei \emph{bekannter} Varianz $\sigma^2$. Es seien also $X_1,X_2,...,X_n \sim N(\mu,\sigma^2)\ i.i.d.$. Wir wollen die Nullhypothese $\mu=\mu_0$ testen.

\vspace{10pt}

\textbf{Nullhypothese und Alternativhypothese:}
\begin{enumerate}[a)]
\item $H_0: \mu=\mu_0$ gegen $H_A:\mu\neq \mu_0$ (zweiseitig)
\item $H_0: \mu=\mu_0$ gegen $H_A:\mu > \mu_0$ (einseitig)
\item $H_0: \mu=\mu_0$ gegen $H_A:\mu < \mu_0$ (einseitig)
\end{enumerate}

\vspace{10pt}

\textbf{Teststatistik:}
\[
\begin{array}{ll}
	\text{Mittelwert $\overline{X}$:} & \overline{X}=\frac{X_1+X_2+...+X_n}{n} \\
	\text{Teststatistik $T$:} & T = \frac{\overline{X}-\mu_0}{\sigma/\sqrt{n}}\sim N(0,1) \\
\end{array}
\]

\vspace{10pt}

\textbf{Verwerfungsbereich:}
\begin{enumerate}[a)]
\item $|T|>t_{1-\frac{\alpha}{2}}=\Phi^{-1}(1-\frac{\alpha}{2})=c$ \\ 
      ($P_{\vartheta_0}[|T|]>\Phi^{-1}(1-\frac{\alpha}{2}]=\alpha$) \\
      Verwerfungsbereich $K=(-\infty,-c) \cup (c,\infty)$
\item $T>\Phi^{-1}(1-\alpha)=c$ \\
      Verwerfungsbereich $K=(c,\infty)$
\item $T<t_{\alpha}=-t_{1-\alpha}=-\Phi^{-1}(1-\alpha)=c$ \\
      Verwerfungsbereich $K=(-\infty,-c)$
\end{enumerate}

\subsubsection{t-Test}
Der t-Test ist ein Test für den Erwartungswert bei \emph{unbekannter Varianz} $\sigma^2$. Es seien also $X_1,X_2,...,X_n \sim N(\mu,\sigma^2) i.i.d.$. Wir wollen die Nullhypothese $\mu=\mu_0$ testen.

\vspace{10pt}

\textbf{Nullhypothese und Alternativhypothese:}
\begin{enumerate}[a)]
\item $H_0: \mu=\mu_0$ gegen $H_A:\mu\neq \mu_0$ (zweiseitig)
\item $H_0: \mu=\mu_0$ gegen $H_A:\mu > \mu_0$ (einseitig)
\item $H_0: \mu=\mu_0$ gegen $H_A:\mu < \mu_0$ (einseitig)
\end{enumerate}

\vspace{10pt}

\textbf{Teststatistik:}
\[
\begin{array}{ll}
	\text{Mittelwert $\overline{X}$:} & \overline{X}=\frac{X_1+X_2+...+X_n}{n} \\
	\text{Schätzfunktion für $S^2$:} & S^2 = \frac{1}{n-1}\cdot \sum\limits_{i=1}^{n}(X_i - \overline{X})^2 \\
	\text{Teststatistik $T$:} & T = \frac{\overline{X}-\mu_0}{S/\sqrt{n}}\sim t_{n-1} \\
\end{array}
\]

Die Teststatistik $T$ ist \textbf{t-verteilt} mit \textbf{$n-1$ Freiheitsgraden}.

\vspace{10pt}

\textbf{Verwerfungsbereich:}
\begin{enumerate}[a)]
\item $|T|>t_{n-1,1-\frac{\alpha}{2}}=c$ \\
      Verwerfungsbereich $K=(-\infty,-c) \cup (c,\infty)$
\item $T>t_{n-1,1-\alpha}=c$ \\
      Verwerfungsbereich $K=(c,\infty)$
\item $T<-t_{n-1,1-\alpha}=-c$ \\
      Verwerfungsbereich $K=(-\infty,-c)$
\end{enumerate}





\subsubsection{Likelihood-Quotienten-Test}
Der Likelihood-Quotienten-Test kann man verwenden, um eine geeignete Teststatistik $T$ zu erhalten. 

\vspace{10pt}

\textbf{Vorgehen:}
\begin{enumerate}
\item Verallgemeinerter Likelihood-Quotient aufstellen:
\[
R(x_1,x_2,...,x_n;\vartheta_0,\vartheta_A):=\frac{L(x_1,x_2,...,x_n;\vartheta_0)}{L(x_1,x_2,...,x_n;\vartheta_A)}
\]
\item Formel vereinfachen. Im Exponent muss eine Summenformel stehen.
\item Überlegen, welcher Wert grösser ist, $\vartheta_0$ oder $\vartheta_A$?                  
\end{enumerate}

Ist dieser Quotient klein, sind die Beobachtungen für die Alternativhypothese deutlich wahrscheinlicher als für die Nullhypothese. Der Verwerfungsbereich $K:=[0,c)$ wird so gewählt, dass der Test das gewünschte Signifikanzniveau einhält.

\subsubsection{Ungepaarter Zweistichproben-z-Test}
Seien $X_1,X_2,...,X_n \sim N(\mu_X,\sigma^2)$ und $Y_1,Y_2,...,Y_m \sim N(\mu_Y,\sigma^2)$ zwei Stichproben mit $m = n$ oder $m \neq n$. Die Erwartungswerte $\mu_X$ und $\mu_Y$ seien unbekannt, die Varianz $\sigma^2$ sei \emph{bekannt}. Der Test kann wie ein normaler z-Test durchgeführt werden, als Teststatistik $T$ wird jedoch folgende Formel verwendet:

\vspace{10pt}

Falls $m\neq n$:

\[
T = \frac{\left(\overline{X}_n-\overline{Y}_m\right)-\left(\mu_X-\mu_Y\right)}{\sigma\sqrt{\frac{1}{n}+\frac{1}{m}}}\sim N(0,1)
\]

Falls $m=n$:

\[
T = \frac{\left(\overline{X}_n-\overline{Y}_m\right)-\left(\mu_X-\mu_Y\right)}{\sigma\sqrt{\frac{2}{n}}}\sim N(0,1)
\]

\subsubsection{Ungepaarter Zweichstichproben-t-Test}
Seien $X_1,X_2,...,X_n \sim N(\mu_X,\sigma^2)$ und $Y_1,Y_2,...,Y_m \sim N(\mu_Y,\sigma^2)$ zwei Stichproben mit $m = n$ oder $m \neq n$. Die Erwartungswerte $\mu_X$ und $\mu_Y$ seien unbekannt, die Varianz $\sigma^2$ sei ebenfalls \emph{unbekannt}. Der Test kann wie ein normaler t-Test durchgeführt werden, als Teststatistik $T$ wird jedoch folgende Formel verwendet:

\[
S_{X}^{2}=\frac{1}{n-1}\sum\limits_{i=1}^{n}\left(X_i-\overline{X}_n\right)^2
\]
\[
S_{Y}^{2}=\frac{1}{m-1}\sum\limits_{j=1}^{m}\left(Y_j-\overline{Y}_m\right)^2
\]

Falls $m\neq n$:

\[
S^2=\frac{1}{m+n-2}\left((n-1)S_{X}^{2}+(m-1)S_{Y}^{2}\right)
\]
\[
T = \frac{\left(\overline{X}_n-\overline{Y}_m\right)-\left(\mu_X-\mu_Y\right)}{S\sqrt{\frac{1}{n}+\frac{1}{m}}}\sim t_{n+m-2}
\]

Falls $m=n$:

\[
S^2=\frac{1}{2(n-1)}(n-1)\left(S_{X}^{2}+S_{Y}^{2}\right)
\]
\[
T = \frac{\left(\overline{X}_n-\overline{Y}_m\right)-\left(\mu_X-\mu_Y\right)}{S\sqrt{\frac{2}{n}}}\sim t_{2n-2}
\]

\subsubsection{Gepaarter Zweistichproben-Test}
In diesem Fall können die beiden Stichproben $X_1,X_2,...,X_n\ i.i.d\sim N(\mu_X,\sigma_X^2)$ und $Y_1,Y_2,...,Y_n\ i.i.d\sim N(\mu_Y,\sigma_Y^2)$ durch Definition einer neuen Zufallsvariable $Z=X_i-Y_i$ auf eine Stichprobe vereinfacht werden. Der Test kann dann wie ein normaler z-Test oder t-Test durchgeführt werden, wobei $Z_1,Z_2,...,Z_n\ i.i.d\sim N(\mu_X-\mu_Y,2\sigma^2)$ gilt.

\subsection{Konfidenzbereiche}
Ein Konfidenzbereich gibt ein Intervall an, in dem sich ein gesuchter Parameter mit sehr hoher Wahrscheinlichkeit befindet.\\

\textbf{Standardabweichung $\sigma$ \emph{bekannt}:}
\[
C(X_1,...,X_n)=
\]
\[
\left[\overline{X}_n-\Phi^{-1}\left(1-\frac{\alpha}{2}\right)\frac{\sigma}{\sqrt{n}},
\overline{X}_n+\Phi^{-1}\left(1-\frac{\alpha}{2}\right)\frac{\sigma}{\sqrt{n}}\right]
\]
\textbf{Standardabweichung $\sigma$ \emph{nicht bekannt}:}
\[
C(X_1,...,X_n)=
\]
\[
\left[\overline{X}_n-t_{n-1,1-\frac{\alpha}{2}}\frac{S}{\sqrt{n}},
\overline{X}_n+t_{n-1,1-\frac{\alpha}{2}}\frac{S}{\sqrt{n}}\right]
\]