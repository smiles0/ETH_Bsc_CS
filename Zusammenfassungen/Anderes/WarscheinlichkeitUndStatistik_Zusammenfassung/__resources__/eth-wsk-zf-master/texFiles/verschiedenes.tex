\section{Verschiedenes}
\textbf{Binomialkoeffizient} \\
Der Binomialkoeffizient
\[
\left(
\begin{array}{c}
	n \\
	k
\end{array}
\right)
\]
gibt für $n,k\in\mathbb{N}$ an, \textbf{wie viele Möglichkeiten es gibt, $k$ Objekte aus $n$ Objekten auszuwählen}. Damit gibt der Binomialkoeffizient an, wie viele $k$-elementige Teilmengen aus einer $n$-elementigen Menge gebildet werden können (gesprochen: ''k aus n'' oder ''k tief n'').

\vspace{10pt}

Für $k=0$ ist der Binomialkoeffizient $1$:
\[
\left(
\begin{array}{c}
	n \\
	0
\end{array}
\right)=1
\]

Für $k=1$ ist der Binomialkoeffizient $n$:
\[
\left(
\begin{array}{c}
	n \\
	1
\end{array}
\right)=n
\]

Für $k>n$ ist der Binomialkoeffizient stets $0$.

\vspace{10pt}

\textbf{Gammafunktion}
\[
\Gamma(x) = \int\limits_{0}^{\infty}t^{x-1}e^{-t}dt=(x-1)!
\]

\vspace{10pt}

\textbf{Diverse Summenformeln}
\[
\sum\limits_{k=0}^{\infty}\frac{1}{k!}=e
\]
\[
\sum\limits_{k=0}^{\infty}\frac{\lambda^k}{k!}=e^{\lambda}
\]
\[
\sum\limits_{k=1}^{\infty}\frac{\lambda^k}{k!}=e^{\lambda}-1
\]
\[
\sum\limits_{i=0}^{n}i=\sum\limits_{i=1}^{n}i=\frac{n(n-1)}{2}
\]
\[
\sum\limits_{i=1}^{n}i^2=\frac{n\cdot (n+1)\cdot (2n+1)}{6}
\]
\[
\sum\limits_{i=0}^{\infty}a\cdot p^i=\sum\limits_{i=1}^{\infty}a\cdot p^{i-1}=\frac{a}{1-p}
\]

\vspace{10pt}

\textbf{Mitternachtsformel} \\
Formel zum Auflösen von allgemeinen quadratischen Gleichungen der Form $ax^2+bx+c=0$:
\[
x_{1,2} = \frac{-b\pm\sqrt{b^2-4ac}}{2a} 
\]

\textbf{Produkteformeln}
\[
\prod\limits_{k=m}^{n}a\cdot x_i = a^{n-m+1}\cdot\prod\limits_{k=m}^{n}x_i
\]

\vspace{10pt}

\textbf{Logarithmenregeln}
\[
\begin{array}{rcl}
	\log{(uv)} & = & \log{(u)}+\log{(v)} \\
	\log{(\frac{u}{v})} & = & \log{(u)}-\log{(v)} \\
	\log_b{(r)} & = & \frac{\log_a(r)}{\log_a{(b)}} \\
	\log_a{(u^k)} & = & k\cdot\log_a{(u)} \\
	\log_a{(\sqrt[n]{u})} & = & \log_a{(u^{\frac{1}{n}})}=\frac{1}{n}\cdot\log_a{(u)} \\
	\log_a{(a^b)} & = & b \\ 
	\log{1} & = & 0 \\
	\log_a{(a)} & = & 1 \\	
	\ln{e} & = & 1 \\
		
\end{array}
\]