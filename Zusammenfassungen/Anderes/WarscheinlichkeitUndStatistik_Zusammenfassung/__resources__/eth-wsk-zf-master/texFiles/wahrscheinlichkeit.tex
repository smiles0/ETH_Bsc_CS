\section{Wahrscheinlichkeit}

\subsection{Allgemeine Rechenregeln}
\[
\begin{array}{rcl}
	P\left[A^C\right] & = & 1-P[A] \\
	P\left[\Omega\right] & = & P[A]+P\left[A^C\right]=1 \\
	P[\emptyset] & = & 0 \\
	P\left[B^C|A\right] & = & 1-P[B|A] \\
	A\subseteq B & \Rightarrow & P[A] \leq P[B] \\
\end{array}
\]

\subsection{DeMorgan'sche Gesetze}
\[
\begin{array}{rcl}
	\overline{A\cap B} & = & \overline{A}\cup\overline{B} \\
	\overline{A\cup B} & = & \overline{A}\cap\overline{B}\\
\end{array}
\]

\subsection{Additionssatz}
Zur Berechnung der Wahrscheinlichkeit, dass Ereignis $A$ oder $B$ eintritt.

\vspace{10pt}

Allgemein:
\[
P[A \cup B] = P[A] + P[B] - P[A \cap B]
\]

A,B disjunkt:
\[
P[A \cup B] = P[A] + P[B]
\]

\textbf{Bemerkungen:}
\begin{itemize}
\item Wenn die Ereignisse nicht offensichtlich disjunkt sind, die \textbf{erste Formel verwenden}!
\end{itemize}

\subsection{Multiplikationssatz}
Zur Berechnung der Wahrscheinlichkeit, dass die Ereignisse $A$ und $B$ eintreten.
\[
P[A \cap B] = P[A] \cdot P[B|A]
\]

\textbf{Bemerkungen}
\begin{itemize}
\item $P[A,B]=P[A \cap B]$.
\end{itemize}

\subsection{Bedingte Wahrscheinlichkeit}
Die bedingte Wahrscheinlichkeit von B gegeben A entspricht der Wahrscheinlichkeit, dass B eintritt, wenn man schon weiss, dass A eingetreten ist. Es gilt:
\[
P[B|A]=\frac{P[A\cap B]}{P[A]}
\]
\[
P[A\cap B]=P[A]\cdot P[B|A]=P[B]\cdot P[A|B]
\]

\textbf{Bemerkungen:}
\begin{itemize}
\item Die \textbf{Pfadregel}, nach der Wahrscheinlichkeiten in einem \textbf{Wahrscheinlichkeitsbaum} multipliziert werden, um die Wahrscheinlichkeit eines Blattes zu erhalten, entspricht einer Verkettung bedingter Wahrscheinlichkeiten.
\end{itemize}

\subsubsection{Satz der totalen Wahrscheinlichkeit}
\[
P[B] = \sum \limits_{i=1}^{n} P[A_i] \cdot P[B|A_i]
\]

\subsubsection{Satz von Bayes}
Zur Berechnung einer bestimmten Zwischenstation $A_k$ in einem Ereignisbaum, wobei mehrere Ereignisse $A_i$ zu Ereignis $B$ führen.
\[
P[A_k|B] = \frac{P[A_k] \cdot P[B|A_k]}{\sum \limits_{i=1}^{n} P[A_i] \cdot P[B|A_i]}
\]

\subsection{Eigenschaften von Ereignissen}

\subsubsection{Unabhängigkeit}
Wenn zwischen zwei Ereignissen $A$ und $B$ kein kausaler Zusammenhang besteht (d.h. es gibt keine gemeinsamen Ursachen oder Ausschliessungen), dann sind sie \emph{unabhängig} voneinander. In diesem Fall gilt:
\[
P[A \cap B] = P[A] \cdot P[B]
\]