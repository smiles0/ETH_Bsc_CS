\setcounter{section}{0} % not strictly necessary, but sometimes useful

\section{Wahrscheinlichkeiten}

\subsection{Grundbegriffe}
\begin{definition}[\textbf{Ereignisraum}]
\textit{Ereignisraum} oder \textit{Grundraum} $\bs{\Omega} \neq \emptyset$ ist Menge aller möglichen Ergebnisse des Zufallsexperiments. Seine Elemente $w\in \Omega$ heissen \textit{Elementarereignisse}.
\end{definition} 

\begin{definition}[\textbf{Potenzmenge, Ereignis}]
Die \textit{Potenzmenge} von $\Omega$ wird mit $2^\Omega$ oder mit $\mathcal{P}(\Omega)$ bezeichnet und ist die Menge aller Teilmengen von $\Omega$. Ein \textit{Ereignis} ist ein solches Element der Potenzmenge, also $A\in\mathcal{P}(\Omega)$. Die Klasse aller beobachtbaren Ereignisse ist $\mathcal{F}$, eine Teilmenge der Potenzmenge.
\end{definition}

\begin{definition}[$\bs{\sigma}$\textbf{-Algebra}]
Ein Mengensystem $\mathcal{F}$ ist eine $\sigma$-Algebra, falls
\begin{itemize}
\item[(i)] $\Omega \in \mathcal{F}$
\item[(ii)] für jedes $A\in\mathcal{F}$ ist auch Komplement $A^\complement \in \mathcal{F}$. 
\item[(iii)] für jede Folge $(A_n)_{n\in\N}$ mit $A_n \in \mathcal{F}$ für alle $n\in \N$ ist auch $\bigcup_{n=1}^\infty A_n \in \mathcal{F}$.
\end{itemize}
\end{definition}

\begin{definition}[\textbf{Wahrscheinlichkeitsmass}]
Ein \textit{Wahrscheinlichkeitsmass} ist eine Abbildung $P: \mathcal{F}\to [0,1]$ mit folgenden Axiomen:
\begin{itemize}
\item[A0)] $P[A] \geq 0 \quad \forall A\in \mathcal{F}$
\item[A1)] $P[\Omega] = 1$
\item[A2)] $P\left[\bigcup_{i=1}^\infty A_i\right] = \sum_{i=1}^\infty P[A_i]$ für disjunkte Ereignisse $A_i$.
\end{itemize}
\end{definition}
Aus den Axiomen A1 und A2 lassen sich die folgenden Rechenregeln herleiten:
\begin{itemize}
\item $P[A^\complement] = 1 - P[A]$
\item $P[\emptyset] = 0$ und $P[\Omega] = 1$
\item $A \subseteq B \implies P[A] \leq P[B]$
\item $P[A \cup B] = P[A] + P[B] - P[A\cap B]$
\end{itemize}

\subsection{Diskrete Wahrscheinlichkeitsräume}
\underline{Annahme:} $\Omega$ ist \textbf{endlich} oder \textbf{abzählbar unendlich} und $\mathcal{F}=2^\Omega$. Hier kann man das Wahrscheinlichkeitsmass definieren, in dem man die Wahrscheinlichkeiten der Elementarereignisse addiert.\\

Ist $\Omega = \{\omega_1, \dots, \omega_N\}$ endlich mit $|\Omega| = N$ und sind alle $\omega_i$ gleich wahrscheinlich, also $p_i = 1/N$, so nennt man $\Omega$ einen \textbf{Laplace Raum} und $P$ ist die \textit{diskrete Gleichverteilung}. Die Wahrscheinlichkeit eines Ereignisses kann dann wie folgt berechnet werden:

$$ P[A] = \frac{\mbox{Anz. Elementarereignisse in } A}{\mbox{Anz. Elementarereignisse in } \Omega} = \frac{|A|}{|\Omega|}$$

\subsection{Bedingte Wahrscheinlichkeiten}
\begin{definition}[\textbf{Bedingte Wahrscheinlichkeit}]
$A,B$ Ereignisse und $P[A] > 0$. Die \textit{bedingte Wahrscheinlichkeit} von $B$ unter der Bedingung $A$ ist definiert als
$$ P[B \with A] := \frac{P[B \cap A ]}{P[A]}$$
Bei fixierter Bedingung $A$ ist $P[\cdot \with A]$ wieder ein Wahrscheinlichkeitsmass auf $(\Omega, \mathcal{F})$.
\end{definition}
$\implies$ \textbf{Multiplikationsregel:} $P[A\cap B] = P[B \with A] \cdot P[A]$ und \textit{Additionsregel:} $P[A\cup B] = P[A] + P[B] - P[A\cap B]$

\begin{satz}[\textbf{Satz der totalen Wahrscheinlichkeit}]
Sei $A_1,\dots,A_n$ eine Zerlegung von $\Omega$ in paarweise disjunkte Ereignisse, d.h. $\bigcup_{i=1}^n A_i = \Omega$ und $A_i \cap A_k = \emptyset \: \forall i\neq k$. Dann gilt:
$$ P[B] = \sum_{i=1}^n P[B \with A_i] \cdot P[A_i]$$
\end{satz}
\begin{proof}
Da $B\subseteq \Omega \implies B \cap \Omega = B = B \cap \left( \bigcup_{i=1}^n A_i \right) = \bigcup_{i=1}^n \left(B \cap A_i \right)$. Weiter sind alle Mengen der Art $(B \cap A_i)$ paarweise disjunkt, was bedeutet, dass $(B\cap A_i)$ eine disjunkte Zerlegung von $B$ bilden. Damit folgt dann 
$$ P[B] = P\left[ \bigcup_{i=1}^n (B \cap A_i) \right] = \sum_{i=1}^n P[B \cap A_i] = \sum_{i=1}^n P[B \with A_i] \cdot P[A_i]$$
\end{proof}
Bedingte Wahrscheinlichkeiten in mehrstufigen Experimenten können oft als Wahrscheinlichkeitsbäume dargestellt werden.

\begin{satz}[\textbf{Satz von Bayes}]
Sei $A_1,\dots,A_n$ eine Zerlegung von $\Omega$ mit $P[A_i] > 0$ für $i = 1 \dots n$ und $B$ ein Ereignis mit $P[B] > 0$, dann gilt für jedes $k$
$$ P[A_k \with B] = \frac{P[B \with A_k]\cdot P[A_k]}{\sum_{i=1}^n P[B \with A_i] \cdot P[A_i]}$$
\[
	\text{einfacher: } P[A\mid B] =
	\frac{P[A \cap B]}{P[B]} =
	\frac{P[B\mid A]\cdot P[A]}{P[B\mid A]\cdot P[A] + P[B \mid \overline{A}]\cdot P[\overline{A}]}
\]
\end{satz}
\begin{proof}
Verwende Definition der bedingten Wahrscheinlichkeit, wende im Zähler die Multiplikationsregel und im Nenner den Satz der totalen Wahrscheinlichkeit an.
\end{proof}

\subsection{Unabhängigkeit}
\begin{definition}[\textbf{Unabhängigkeit von 2 Ereignissen}]
Zwei Ereignisse $A,B$ heissen \textit{stochastisch unabhängig} falls $P[A \cap B] = P[A] \cdot P[B]$. Ist $P[A]=0$ oder $P[B] = 0$, so sind zwei Ereignisse immer unabhängig. Ist $P[A]\neq 0$, dann gilt folgende Äquivalenz:
$$ A, B \mbox{ sind unabhängig } \LLRA P[B \with A] = P[B]$$
Analoges gilt falls $P[B] \neq 0$.
\end{definition}

\begin{definition}[\textbf{allgemeine Unabhängigkeit}]
Ereignisse $A_1,\dots,A_n$ heissen \textit{stochastisch unabhängig}, falls für jede endliche Teilfamilie die Produktformel gilt. D.h. für ein $m \in \N$ und $\{k_1,\dots, k_m\} \subseteq \{1, \dots, n\}$ gilt immer
$$ P \left[ \bigcap_{i=1}^m A_{k_i} \right] = \prod_{i=1}^m P[A_{k_i}]$$
\end{definition}


	





	
	
	
	