
% ------------------------------------------------------------------------------------------------ %
% DESKRIPTIVE STATISTIK
% ------------------------------------------------------------------------------------------------ %


\section{Deskriptive Statistik}


% ------------------------------------------------------------------------------------------------ %
% HISTOGRAMM
% ------------------------------------------------------------------------------------------------ %


\subsection{Histogramm}

Bei grossem Stichprobenumfang \(n\) werden benachbarte Werte zu einer Klasse zusammengefasst. Der Wertebereich der Daten wird dadurch in disjunkte Intervalle (die Klassen) unterteilt.
\begin{compactenum}[i)]
	\item Die Anzahl der Klassen sollte von der Grössenordnung \(\sqrt{n}\) sein.
	\item Die Klassenbreite sollte für alle Klassen gleich sein; als Ausnahme können die Klassen am linken und rechten Rand grösser sein (Ausreisser).
\end{compactenum}

\begin{figure}[htb]
	\begin{center}
		\scalebox{0.6}[0.6]{
			\includegraphics{figures/hist.eps}}
	\end{center}
	\vspace{-2em}
	\caption{Histogramm einer standard-norvmalverteilten Zufallsvariable.}
\end{figure}


% ------------------------------------------------------------------------------------------------ %
% BOXPLOT
% ------------------------------------------------------------------------------------------------ %


\subsection{Boxplot}

Aus einem Boxplot lässt sich folgendes ablesen:
\begin{compactenum}[a:]
	\item empirischer Median
	\item empirisches \(0.25\)-Quantil
	\item empirisches \(0.75\)-Quantil
	\item kleinster Datenwert \(x_i\) mit \(b-x_i < 1.5 (c-b)\)
	\item grösster Datenwert \(x_i\) mit \(x_i - c < 1.5 (c-b)\)
	\item Ausreisser
\end{compactenum}

\begin{figure}[htb]
	\begin{center}
		\scalebox{0.6}[0.6]{
			\includegraphics{figures/boxplot.eps}}
	\end{center}
	\vspace{-3em}
	\caption{Boxplot}
\end{figure}
%a: empirischer Median, b: empirisches \(0.25\)-Quantil, c: empirisches \(0.75\)-Quantil, d: kleinster Datenwert \(x_i\) mit \(b-x_i < 1.5 (c-b)\), e: grösster Datenwert \(x_i\) mit \(x_i - c < 1.5 (c-b)\).


% ------------------------------------------------------------------------------------------------ %
% QQ-PLOT
% ------------------------------------------------------------------------------------------------ %


\subsection{QQ-Plot}

Mit einem \emph{QQ-Plot} (Quantil-Quantil) kann man die Abweichung der Daten von einer gewählten Modell-Verteilung \(F\) graphisch überprüfen.

Es werden die empirischen Quantile auf der \(y\)-Achse gegenüber den theoretischen Quantilen auf der \(x\)-Achse geplottet.
