
% ------------------------------------------------------------------------------------------------ %
% TESTS
% ------------------------------------------------------------------------------------------------ %


\section{Tests}

\subsection{Grundlagen}

Ausgangspunkt ist eine Stichprobe \(X_1,\ldots,X_n\) in einem Modell \(\P_\theta\) mit unbekanntem
Parameter \(\theta \in \Theta\).


\begin{definition}[Modellklassen]
	Aufgrund einer Vermutung, wo sich der richtige Parameter \(\theta\) befindet,
	werden eine \emph{Hypothese} \(\Theta_0 \subseteq \Theta\) und eine \emph{Alternative}
	\(\Theta_A \subseteq \Theta\) mit \(\Theta_0 \cap \Theta_A = \emptyset\) formuliert:

	\begin{center}
		\begin{tabular}{r@{ : }l}
			Hypothese \(H_0\)   & \(\theta \in \Theta_0\) \\
			Alternative \(H_A\) & \(\theta \in \Theta_A\)
		\end{tabular}
	\end{center}
\end{definition}

\begin{note}
	Die Hypothese (bzw. Alternative) heisst \emph{einfach},
	falls sie nur aus einem einzelnen Wert besteht, z.B. \(\Theta_0 = \{\theta_0\}\)
\end{note}

\begin{definition}[Test]
	Eine Interpretation der Übereinstimmung zwischen den Daten
	\(x_1, \ldots, x_n\) und einem vermutetem Modell.

	\begin{compactenum}
		\item \emph{Entscheidungsregel:} \(\1_{t(x_1,\ldots,x_n) \in K}\)
		\item \emph{Teststatistik:} \(T = t (X_1,\ldots,X_n)\)
		\item \emph{Verwerfungsbereich:} \(K \subseteq \R\)
	\end{compactenum}
\end{definition}

\begin{definition}[Fehler 1. Art] Die Hypothese wird abgelehnt, obwohl sie richtig ist.
	Die Wahrscheinlichkeit für einen Fehler 1. Art ist
	\[
		\P_\theta[T \in K]
		\quad\text{für } \theta \in \Theta_0.
	\]
\end{definition}

\begin{definition}[Fehler 2. Art]
	Die Hypothese \(\theta\) wird akzeptiert, obwohl sie falsch ist.
	Die Wahrscheinlichkeit für einen Fehler 2. Art ist
	\[
		\P_\theta[T \notin K] =
		1 - \P_\theta[T \in K]
		\quad\text{für } \theta \in \Theta_A.
	\]
\end{definition}


% ------------------------------------------------------------------------------------------------ %
% MÖGLICHES VORGEHEN
% ------------------------------------------------------------------------------------------------ %


\subsection{Konstruktion von Tests \((T,K)\)}

\begin{compactenum}
	\item
	Wähle eine \emph{Hypothese} \(\Theta_0\) (normalerweise \(\{\theta_0\} = \Theta_0\)).

	\item
	Wähle ein \emph{Signifikanzniveau} \(\alpha \in (0,1)\).
	Dadurch werden Fehler 1. Art von \(\alpha\) kontrolliert.
	\[	\sup_{\theta \in \Theta_0} \P_\theta[T \in K] \leq \alpha.	\]

	\item
	Maximiere die \emph{Macht} \(\beta\) des Tests.
	Dadurch werden Fehler 2. Art minimiert.
	\[ \beta(\theta) := \mathbb{P}_\theta[T \in K] \qquad \text{für } \theta \in \Theta_0 \]

	\item
	Falls der realisierte Wert \(T(\omega)\) im Verwerfungsbereich \(K_\alpha\) liegt,
	wird die Nullhypothese auf dem Niveau \(\alpha\) verworfen
\end{compactenum}

\begin{note}
	Die Hypothese zu verwerfen, ist schwieriger. Es
	wird häufig die Negation der gewünschten Aussage benutzt.
\end{note}

\begin{definition}[P-Wert]
	Der P-Wert ist die Wahrscheinlichkeit, dass unter der Nullhypothese \(H_0\) ein zufälliger Versuch
	mindestens so extrem ausfällt, wie der beobachtete Wert \(T(\omega)\).
	\[
		\textbf{p-Wert} =
		\left\{\begin{array}{cl}
			\P_{\theta_0}[T \leq T(\omega)] & \text{für\quad} H_A: \theta < \theta_0 \\
			\P_{\theta_0}[T \geq T(\omega)] & \text{für\quad} H_A: \theta > \theta_0
		\end{array}\right.
	\]
\end{definition}



% ------------------------------------------------------------------------------------------------ %
% LIKELIHOOD QUOTIENT
% ------------------------------------------------------------------------------------------------ %

\begin{definition}[Likelihood-Quotienten Test]
	Als Teststatistik wird der \emph{Likelihood-Quotient} \(\mathcal{R}\) gewählt,
	\[
		T :=
		\mathcal{R}(x_1,\ldots,x_n) :=
		\frac
		{\displaystyle\sup_{\theta\in\Theta_0}\mathcal{L}(x_1,\ldots,x_n ; \theta)}
		{\displaystyle\sup_{\theta\in\Theta_A}\mathcal{L}(x_1,\ldots,x_n ; \theta)}
	\]
	Ist dieser Quotient klein, sind die Beobachtungen im Modell \(\P_{\Theta_A}\) deutlich wahrscheinlicher
	als im Modell \(\P_{\Theta_0}\). Der Verwerfungsbereich \(K := [0,c)\) wird so gewählt,
	dass der Test das gewünschte Signifikanzniveau einhält.
\end{definition}

\begin{theorem}[Neyman-Pearson]
	Sind Hypothese und Alternative beide einfach, so ist der Test optimal bzgl. Fehler 1. \& 2. Art
\end{theorem}


% ------------------------------------------------------------------------------------------------ %
% Z-TEST
% ------------------------------------------------------------------------------------------------ %

\subsection{Einstichprobentests}
Neyman-Pearson gibt uns eine Systematische Methode um 'gute' Tests zu konstruieren.
Hier ein paar Beispiele:

\begin{definition}[z-Test] Test für Erwartungswert bei bekannter Varianz \(\sigma^2\)

	\begin{compactenum}
		\item Stichproben \(X_1, \ldots, X_n \iid \Stdev(\theta,\sigma^2)\)
		\item Nullhypothese \(H_0: \theta = \theta_0\)
		\item Teststatistik: \( T := \frac{\overline{X}_n - \theta_0}{\sigma / \sqrt{n}} \sim \Stdev(0,1)\) unter \(\P_{\theta_0}\)
		\item Signifikanzniveau: wir wählen ein \(\alpha \in (0,1)\)
		\item Wir konstruieren den Verwerfungsbereich \(K_\alpha\)
		\[
			K_\alpha =	\left\{\begin{array}{cl}
				(-\infty, z_{\alpha})                                 & \text{für\quad} H_A: \theta < \theta_0    \\
				(z_{1-\alpha}, \infty)                                & \text{für\quad} H_A: \theta > \theta_0    \\
				(-\infty, z_{\alpha/2}) \cup (z_{1-\alpha/2}, \infty) & \text{für\quad} H_A: \theta \neq \theta_0
			\end{array}\right.
		\]
		\item Falls \(T(\omega) \in K_\alpha\), wird \(H_0\) auf dem Niveau \(\alpha\) verworfen
	\end{compactenum}
\end{definition}

%\begin{example}
%	Zum Beispiel liefert die Bedingung
%	\[
%		\alpha =
%		\P_{\theta_0}[T \in K_>] =
%		\P_{\theta_0}[T > c_>] =
%		1- \Phi(c_>),
%	\]
%	dass \(c_> = \Phi^{-1}(1-\alpha)\), also das \((1-\alpha)\)-Quantil der \(\mathcal{N}(0,1)\)-Verteilung, sein muss.
%\end{example}


% ------------------------------------------------------------------------------------------------ %
% T-TEST
% ------------------------------------------------------------------------------------------------ %

\begin{definition}[t-Test] Test für Erwartungswert bei unbekannter Varianz \(\sigma^2\)

	Die Varianz \(S^2\) wird geschätzt
	\begin{compactenum}
		\item Stichproben \(X_1, \ldots, X_n \iid \Stdev(\mu,\sigma^2)\)
		\item Nullhypothese \(H_0: \mu = \mu_0\)
		\item Teststatistik: \( T := \frac{\overline{X}_n - \mu_0}{S / \sqrt{n}} \sim t_{n-1}\) unter \(\P_{\mu_0}\)
		\item Signifikanzniveau: wir wählen ein \(\alpha \in (0,1)\)
		\item Wir konstruieren den Verwerfungsbereich \(K_\alpha\)
		\[
			K_\alpha
			=	\left\{\begin{array}{cl}
				(-\infty, t_{n-1,\alpha})                                     & \text{für } \mu < \mu_0    \\
				(t_{n-1,1-\alpha}, \infty)                                    & \text{für } \mu > \mu_0    \\
				(-\infty, t_{n-1,\alpha/2}) \cup (t_{n-1,1-\alpha/2}, \infty) & \text{für } \mu \neq \mu_0
			\end{array}\right.
		\]
		\item Falls \(T(\omega) \in K_\alpha\), wird \(H_0\) auf dem Niveau \(\alpha\) verworfen
	\end{compactenum}
\end{definition}


% ------------------------------------------------------------------------------------------------ %
% GEPAARTE ZWEISTICHPROBEN TESTS
% ------------------------------------------------------------------------------------------------ %

\subsection{Zweistichprobentest}

\begin{definition}[gepaart] Seien \((X_1, Y_1) \ldots (X_n,Y_n)\) natürliche Paare von ZV
	mit \(X_1, \ldots, X_n \iid \Stdev(\mu_X,\sigma^2)\)
	und \(Y_1, \ldots, Y_n \iid \Stdev(\mu_Y,\sigma^2)\)

	Dann lässt sich der Test zum Vergleich von \(\mu_X\) und \(\mu_Y\) auf eine Stichprobe zurückführen:
	\[	Z_i := X_i - Y_i \iid \Stdev(\mu_x-\mu_y,2\sigma^2) 	\]
\end{definition}


\begin{definition}[ungepaart] Test mit bekannten Varianzen \(\sigma_X, \sigma_Y > 0\)

	\begin{compactenum}
		\item Stichproben \\
		\(X_1, \ldots, X_n \iid \Stdev(\mu_X,\sigma_X^2)\),\quad
		\(Y_1, \ldots, Y_m \iid \Stdev(\mu_Y,\sigma_Y^2)\)
		\item Nullhypothese \(H_0: \mu_X - \mu_Y = \mu_0\)
		\item Teststatistik: \( T := \frac{ \overline{X}_n - \overline{Y}_n - \mu_0 }
		{\sqrt{ \frac{\sigma_X^2}{n} + \frac{\sigma_Y^2}{m} }} \sim \Stdev(0,1)\) unter \(\P_{\mu_0}\)
		\item Signifikanzniveau: wir wählen ein \(\alpha \in (0,1)\)
		\item Wir konstruieren den Verwerfungsbereich \(K_\alpha\)
		\[
			K_\alpha
			=	\left\{\begin{array}{cl}
				(-\infty, z_{\alpha})                                 & \text{für } \mu_X - \mu_Y < \mu_0    \\
				(z_{1-\alpha}, \infty)                                & \text{für } \mu_X - \mu_Y > \mu_0    \\
				(-\infty, z_{\alpha/2}) \cup (z_{1-\alpha/2}, \infty) & \text{für } \mu_X - \mu_Y \neq \mu_0
			\end{array}\right.
		\]
		\item Falls \(T(\omega) \in K_\alpha\), wird \(H_0\) auf dem Niveau \(\alpha\) verworfen
	\end{compactenum}
\end{definition}


\begin{definition}[ungepaart] Test bei unbekannter, gleicher Varianz \(\sigma > 0\)

	Die Varianz \(S^2 = \frac{ (n-1)S_X^2 + (m-1)S_Y^2 }{ n + m - 2}\) wird geschätzt
	\begin{compactenum}
		\item Stichproben \\
		\(X_1, \ldots, X_n \iid \Stdev(\mu_X,\sigma^2)\), \quad
		\(Y_1, \ldots, Y_m \iid \Stdev(\mu_Y,\sigma^2)\)
		\item Nullhypothese \(H_0: \mu_X - \mu_Y = \mu_0\)
		\item Teststatistik: \( T := \frac{ \overline{X}_n - \overline{Y}_n - \mu_0 }
		{S \sqrt{ \frac{1}{n} + \frac{1}{m} }} \sim t_{n+m-2}\) unter \(\P_{\mu_0}\)
		\item Signifikanzniveau: wir wählen ein \(\alpha \in (0,1)\)
		\item Wir konstruieren den Verwerfungsbereich \(K_\alpha\)
		\[
			K_\alpha
			=	\left\{\begin{array}{cl}
				(-\infty, t_{n+m-2,\alpha})  & \text{für } \mu_X - \mu_Y < \mu_0    \\
				(t_{n+m-2,1-\alpha}, \infty) & \text{für } \mu_X - \mu_Y > \mu_0    \\
				\left.
				\begin{array}{r}
					(-\infty, t_{n+m-2,\alpha/2}) \\
					\cup (t_{n+m-2,1-\alpha/2}, \infty)
				\end{array}
				\right.
				                             & \text{für } \mu_X - \mu_Y \neq \mu_0
			\end{array}\right.
		\]
		\item Falls \(T(\omega) \in K_\alpha\), wird \(H_0\) auf dem Niveau \(\alpha\) verworfen
	\end{compactenum}
\end{definition}


% ------------------------------------------------------------------------------------------------ %
% VERTRAUENSINTERVALL
% ------------------------------------------------------------------------------------------------ %


\section{Konfidenzbereiche}

\begin{definition}[Konfidenzbereich]
	Ein \emph{Konfidenzbereich} für einen Schäetzwert \(\theta\) von den Stichproben \(X_1,\ldots,X_n\)
	ist eine Menge \(C(X_1,\ldots,X_n) \subseteq \Theta\).
	In den meisten Fällen ist das ein Intervall, dessen Endpunkte von \(X_1,\ldots,X_n\) abhängen.

	\(C\) heisst ein Konfidenzbereich zum \emph{Niveau} \(1-\alpha\), falls gilt
	\[	1 - \alpha \leq \P_\theta[\theta \in C(X_1,\ldots,X_n)] \quad \text{für alle } \theta \in \Theta\]
\end{definition}


\begin{example} Konfidenzintervall des Stichprobenmittels \\
	Annahme: \(C(X_1, \ldots, X_n) = [\, \theta - \delta, \, \theta + \delta \,]\) \\
	\(1 - \alpha
	\, \leq \, \P_\theta\big[\abs{\overline{X}_n - \mu} \leq \delta\big]
	\, \leq \, \P_\theta\big[\abs{ \frac{\overline{X}_n - \mu}{S/\sqrt{n}}} \leq \frac{\delta}{S/\sqrt{n}}\big]\) \\
	Satz 7.1 \( \Rightarrow \delta = t_{ n-1 , 1 - \frac{\alpha}{2} } \cdot \frac{S}{\sqrt{n}}\)
\end{example}

\begin{example} Konfidenzintervall der Stichprobenvarianz \\
	\(1 - \alpha
	= \P_\theta\big[ \chi_{n-1,\frac{\alpha}{2}}^2 \leq \sigma \leq \chi_{n-1,1 - \frac{\alpha}{2}}^2  \big]\) \\
	\(	= \P_\theta\big[ \frac{(n-1)S^2}{\chi_{n-1,\frac{\alpha}{2}}^2} \leq \sigma \leq \frac{(n-1)S^2}{\chi_{n-1,1 - \frac{\alpha}{2}}^2}  \big]
	\Rightarrow C =  \big[ \frac{(n-1)S^2}{\chi_{n-1,\frac{\alpha}{2}}^2}, \frac{(n-1)S^2}{\chi_{n-1,1 - \frac{\alpha}{2}}^2}  \big]\)

\end{example}



% ------------------------------------------------------------------------------------------------ %
