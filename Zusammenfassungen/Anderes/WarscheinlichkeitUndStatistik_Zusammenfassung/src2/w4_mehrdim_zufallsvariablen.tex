
% ------------------------------------------------------------------------------------------------ %
% ZUFALLSVARIABLEN IN R^n
% ------------------------------------------------------------------------------------------------ %


\section{Zufallsvariablen in \(\R^n\)}

\begin{definition}[Verteilung]
	Das stochastische Verhalten einer Zufallsvariablen \\ \(X=(X_1,\ldots,X_n)\)
	über \textbf{einem} Wahrscheinlichkeitsraum wird durch ihre \emph{Verteilung} beschrieben.
	\(\mu_X :\R^n \rightarrow[0,1]\)
	\begin{eqnarray*}
		\begin{split}
			\mu_X(B)	 &:= \P[(X_1,\ldots,X_n) \in B] \\
			&:=  P[\{\omega \mid (X_1(\omega),\ldots,X_n(\omega)) \in B\}]
			\quad \text{für } B \subseteq \R^n
		\end{split}
	\end{eqnarray*}
	\begin{note}

		Jedes Wahrscheinlichkeitsmass \(\mu_X\) erfüllt:
		\begin{compactenum}
			\item \(\mu_X(B) \geq 0 \quad \text{für alle } B \subseteq \R^n \)
			\item \(\mu_X(\mathcal{W}(X)) = \mu_X(\R^n) = 1\)
		\end{compactenum}
	\end{note}

\end{definition}

% ------------------------------------------------------------------------------------------------ %
% GEMEINSAME VERTEILUNGEN
% ------------------------------------------------------------------------------------------------ %

\subsection{Gemeinsame Verteilungen}


\begin{definition}[Gemeinsame Verteilung]
	Die \emph{gemeinsame Verteilungsfunktion} ist die Abbildung \(F_X:\R^n \rightarrow [0,1]\),
	\begin{eqnarray*}
		\begin{split}
			F_X(t_1,\ldots,t_n) &:= \P[X_1 \leq t_1,\ldots,X_n \leq t_n] \\
			&:= \mu_X(  (-\infty, t_1] \times \ldots \times (-\infty, t_n] )
		\end{split}
	\end{eqnarray*}
\end{definition}


\begin{definition}[Gemeinsame Gewichtsfunktion]
	Im diskreten Fall ist die \emph{gemeinsame Gewichtungsfunktion} \(p_X:\R^n\rightarrow [0,1]\) definiert:
	\[		p_X(x_1,\ldots,x_n) := \P[X_1=x_1,\ldots,X_n=x_n]	\]

	Somit lässt sich die \emph{diskrete Verteilung \(\mu_X\)} berechnen:
	\[\mu_X(B) = \sum_{ (x_1, \ldots, x_n)_i \in B } p_X( (x_1, \ldots, x_n)_i ) \]

\end{definition}


\begin{definition}[Gemeinsame Dichte]
	Im stetigen Fall ist die \emph{gemeinsame Dichte} \(f_X : \R^n \rightarrow [0,\infty)\) definiert, falls für alle \(t_i \in \R\) gilt
	\[
		F_X(t_1,\ldots,t_n) = \int_{-\infty}^{t_1} \ldots \int_{-\infty}^{t_n} f_X(x_1,\ldots,x_n) \d x_n \ldots \d x_1
	\]

	Somit lässt sich die \emph{stetige Verteilung \(\mu_X\)} berechnen:
	\[\mu_X(B) = \int_{ (t_1, \ldots, t_n)   \in B } f_X(t_1,\ldots,t_n) \d\mu \]


	\begin{note}
		\(f_X(t_1,\ldots,t_n) = \frac{\partial^n}{\partial t_1 \cdots \partial t_n} F(t_1,\ldots,t_n)\)
	\end{note}

\end{definition}


% ------------------------------------------------------------------------------------------------ %
% RANDVERTEILUNG
% ------------------------------------------------------------------------------------------------ %


\subsection{Randverteilungen}

\begin{definition}[Randverteilung]

	Seien \(X\) und \(Y\) Zufallsvariablen mit gemeinsamer Verteilungsfunktion \(F_{X,Y}\),
	dann ist die \emph{Randverteilung} \(F_X:\R \rightarrow [0,1]\) von \(X\) definiert durch
	\[
		F_X(x) := \P[X \leq x] = \P[X \leq x, Y < \infty] = \lim_{y\rightarrow\infty}F_{X,Y}(x,y).
	\]
\end{definition}

\begin{definition}[Randgewichtsfunktion]

	Für zwei diskrete Zufallsvariablen \(X\) und \(Y\) mit gemeinsamer Gewichtsfunktion \(p_{X,Y}(x,y)\)
	ist die Gewichtsfunktion der Randverteilung von \(X\) gegeben:
	\[
		p_X(x) =
		\P[X=x] =
		\sum_j \P[X=x,Y=y_j] =
		\sum_j p_{X,Y}(x,y_j).
	\]
\end{definition}

\begin{definition}[Randdichte]

	Für zwei stetige Zufallsvariablen \(X\) und \(Y\) mit gemeinsamer Dichte \(f_{X,Y}(x,y)\) ist die Randdichte (Dichtefunktion der Randverteilung) von \(X\) gegeben durch
	\[
		f_X(x) = \int_{-\infty}^\infty f_{X,Y}(x,y) \d y.
	\]
\end{definition}


% ------------------------------------------------------------------------------------------------ %
% BEDINGTE VERTEILUNG
% ------------------------------------------------------------------------------------------------ %


\subsection{Bedingte Verteilung}

\begin{definition}[Bedingte Gewichtsfunktion]
	Seien \(X\) und \(Y\) diskrete Zufallsvariablen mit gemeinsamer Gewichtsfunktion \(p_{X,Y}(x,y)\),
	dann ist die \emph{bedingte Gewichtsfunktion} \(p_{X \mid Y}(x \mid y)\) von \(X\) gegeben \(Y\) definiert durch
	\[
		p_{X \mid Y}(x \mid y) := \P[X = x \mid Y = y] = \frac{p_{X,Y}(x,y)}{p_Y(y)}
	\]
	falls \(p_Y(y) > 0\) und \(0\) falls \(p_Y(y) = 0\).
\end{definition}

\begin{definition}[Bedingte Dichte]
	Für zwei stetige Zufallsvariablen \(X\) und \(Y\) mit gemeinsamer Dichte \(f_{X,Y}(x,y)\) ist
	die \emph{bedingte Dichte} \(f_{X \mid Y}\) von \(X\) gegeben \(Y\) definiert durch
	\[
		f_{X \mid Y}(x \mid y) := \frac{f_{X,Y}(x,y)}{f_Y(y)}
	\]
	falls \(f_Y(y) > 0\) und \(0\) falls \(f_Y(y) = 0\).
\end{definition}


% ------------------------------------------------------------------------------------------------ %
% UNABHÄNGIGKEIT
% ------------------------------------------------------------------------------------------------ %


\subsection{Unabhängigkeit}

\begin{definition}[Unabhängigkeit]

	Die Zufallsvariablen \(X_1,\ldots,X_n\) heissen \emph{unabhängig}, falls gilt:
	\[
		F(x_1,\ldots,x_n) = \prod_{i=1}^n F_{X_i}(x_i)
	\]

\end{definition}

\begin{note}
	Es gelten folgende Rechenregeln:
	\begin{compactenum}[i:]
		\item \(p(x_1,\ldots,x_n) = \prod p_{X_i}(x_i)\)
		\item \(	f(x_1,\ldots,x_n) = \prod f_{X_i}(x_i)\)
		\item \(\mathcal{M}_{(X_1, \ldots, X_n)}(t_1, \ldots, t_n)= \prod \mathcal{M}_{X_i}(t_i)\)
		\item \(Y_i = f_i(X_i)\) sind für beliebige \(f_i\) unabhängig
	\end{compactenum}

\end{note}

\begin{definition}[i.i.d Annahme]
	Die Abkürzung \emph{i.i.d.} kommt vom Englischen \emph{independent and identically distributed}.
\end{definition}

\begin{note}
	Es gelten die Implikationen: \\
	unabhängig	\(\Rightarrow\) paarweise unabhängig \(\Rightarrow\) unkorreliert
\end{note}


% ------------------------------------------------------------------------------------------------ %
% Erwartungswert
% ------------------------------------------------------------------------------------------------ %
\subsection{Erwartungswert und Varianz}

\begin{note}
	Der Erwartungswert einer \(n\)-dimensionalen Verteilung wird
	als \(n\)-Tupel der Erwartungswerte aller Randverteilungen \(\E[X_i]\) angegeben.
\end{note}

\begin{theorem}[4.2]
	Für den Erwartungswert \(\E[Y]\) einer Funktion \(Y := g(X_1, \ldots X_n)\) der Zufallsvariablen \(X_1,\ldots,X_n\) gilt
	\[
		\E[Y] = \sum_{x_1,\ldots,x_n} g(x_1,\ldots,x_n) p(x_1,\ldots,x_n)
	\]
	\[
		\E[Y] = \underset{\R^n}{\int\ldots\int} g(x_1,\ldots,x_n)f(x_1,\ldots,x_n) \d x_n \ldots \d x_1.
	\]
\end{theorem}


% ------------------------------------------------------------------------------------------------ %
% KOVARIANZ UND KORRELATION
% ------------------------------------------------------------------------------------------------ %


\begin{note}

	Es gelten folgende Rechenregeln:
	\begin{compactenum}[i:]

		\item \(\E\left[a+\sum_{i=1}^n b_i X_i\right]=a + \sum_{i=1}^n b_i \E[X_i]\)

		\item \(\E[\prod_{i=1}^n X_i] = \prod_{i=1}^n X_i \E[X_i] \Leftrightarrow X_i\) unabhängig

		\item \(\var[a+\sum_{i=1}^n b_iX_i] = \sum_{i=1}^n b_i^2 \var[X_i]\), für \(X_i\) unabhängig

		\item \(\cov[a+\sum_{i=1}^n b_iX_i,c+\sum_{j=1}^m d_jY_j]\) \\
		\(=\sum_{i=1}^n\sum_{j=1}^m b_i d_j \cov[X_i,Y_j]\)

	\end{compactenum}
\end{note}

% ------------------------------------------------------------------------------------------------ %
